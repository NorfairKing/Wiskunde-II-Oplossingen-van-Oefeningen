\documentclass[11pt]{article}
\usepackage{amsmath,amssymb,hyperref}
\usepackage{ulem}
\title{Oefeningen Wiskunde II 2013, 1e Bach Informtica}
\author{Syd Kerckhove}

\date{\today}
\hoffset=-2,0cm

\begin{document}
\maketitle
\section*{Hoofdstuk 1}
\subsection*{1.1}
\subsubsection*{a)}

\[
\begin{pmatrix}
  1 & 1 & -2 & 10 \\
  4 & 2 & 2 & 1 \\
  6 & 2 & 3 & 4  \\
 \end{pmatrix} 
 \overset{\begin{matrix}
  R2 = R2-4R1 \\
  R3 = R3-6R1
 \end{matrix}}{\rightarrow} 
 \begin{pmatrix}
  1 & 1 & -2 & 10 \\
  0 & -2 & 10 & -39 \\
  0 & -4 & 15 & -56  \\
 \end{pmatrix}
 \overset{\begin{matrix}
  R2 = -\frac{1}{2}R2 \\
  R3 = -\frac{3}{4}R3
 \end{matrix}}{\rightarrow} 
 \begin{pmatrix}
  1 & 1 & -2 & 10 \\
  0 & 1 & -5 & \frac{39}{2} \\
  0 & 1 & -\frac{15}{4} & 14  \\
 \end{pmatrix}
\]
\[ 
\overset{\begin{matrix}
  R1=R1-R2\\
  R3=R3-R2
 \end{matrix}}{\rightarrow} 
 \begin{pmatrix}
  1 & 0 & 3 & -\frac{19}{2} \\
  0 & 1 & 3 & \frac{39}{2} \\
  0 & 0 & \frac{5}{4} & -\frac{11}{2} \\
 \end{pmatrix}
 \overset{\begin{matrix}
  R3 = \frac{4}{5}R3
 \end{matrix}}{\rightarrow} 
 \begin{pmatrix}
  1 & 0 & 3 & -\frac{19}{2} \\
  0 & 1 & 3 & \frac{39}{2} \\
  0 & 0 & 1 & -\frac{22}{5} \\
 \end{pmatrix}
 \overset{\begin{matrix}
  R1 = R1 - 3R3 \\
  R2 = R2 - 3R3
 \end{matrix}}{\rightarrow}
  \begin{pmatrix}
  1 & 0 & 0 & \frac{37}{10} \\
  0 & 1 & 0 & -\frac{5}{2} \\
  0 & 0 & 1 & -\frac{22}{5} \\
 \end{pmatrix}
\]
Antwoord:\\
$ V = \left\{\left(\frac{37}{10}, -\frac{5}{2}, -\frac{22}{5}\right)\right\}$

\subsubsection*{b)}
\[
\left(
\begin{array}{ccc|c}
1 & 2 & -3 & -1 \\
3 & -1 & 2 & 7 \\
5 & 3 & -4 & 2
\end{array}
\right)
\overset{\begin{matrix}
  R2 = R2 - 3R1 \\
  R3 = R3 - 5R1
 \end{matrix}}{\rightarrow}
 \left(
\begin{array}{ccc|c}
1 & 2 & -3 & -1 \\
0 & -7 & 11 & 10 \\
0 & -7 & 11 & 7
\end{array}
\right)
\]
Antwoord:\\
$V = \emptyset$

\subsection*{1.2}
\subsubsection*{a)}
\[
\left(
\begin{array}{ccc|c}
2 & -1 & 1 & 2 \\
3 & 1 & -2 & 1 \\
1 & -3 & 4 & k
\end{array}
\right)
\rightarrow
\left(
\begin{array}{ccc|c}
1 & 0 & -\frac{1}{5} & \frac{3}{5} \\
0 & 1 & -\frac{7}{5} & -\frac{4}{5} \\
0 & 0 & 0 & 0
\end{array}
\right)
\]
Antwoord:\\
$V = \left\lbrace (\frac{3}{5} + \frac{1}{5}\lambda, -\frac{4}{5} + \frac{7}{5}\lambda, \lambda)|\lambda \in \mathbb{R} \right\rbrace$

\subsection*{1.4}
$(Ax+B)(x^2-1) + C(x^2+4)(x+1) + D(x^2+4)(x-1)$\\
$= Ax^3-Ax+Bx^2-B+cx^3+cx^2+4Cx+4C+Dx^3-Dx^2+4Dx-4D$
\[
\left\{ 
  \begin{array}{l l}
    A+C+D=0\\
	B+C-D=0\\
	-A+4C+4D=0\\
	-B+4C-4D=1
  \end{array} \right.
  \longrightarrow
  \left(
\begin{array}{cccc|c}
1 & 0 & 1 & 1 & 0 \\
0 & 1 & 1 & -1 & 0 \\
-1 & 0 & 4 & 4 & 0 \\
0 & -1 & 4 & -4 & 1
\end{array}
\right)
\overset{GRM}{\rightarrow}
\left(
\begin{array}{cccc|c}
1 & 0 & 0 & 0 & 0 \\
0 & 1 & 0 & 0 & -\frac{1}{5} \\
0 & 0 & 1 & 0 & \frac{1}{10} \\
0 & 0 & 0 & 1 & -\frac{1}{10}
\end{array}
\right)
\]
Antwoord:\\
\[
\left\{ 
  \begin{array}{l l}
    A=0\\
	B=-\frac{1}{5}\\
	C= \frac{1}{10}\\
	D=-\frac{1}{10}
  \end{array} \right.
\]

\subsection*{1.5}
\[
\frac{1}{x(x^2-b^2)} = \frac{A}{x} + \frac{B}{x+b} + \frac{C}{x-b}
\]
\[
\Leftrightarrow
\frac{Ax^2-Ab^2+Bx^2+Bxb+Cx^2-Cxb}{x(x^2-b^2)}
\]
\[
\Leftrightarrow
\left\lbrace
\begin{array}{c c}
A+B+C=0\\
Bb-Cb=0\\
-Ab^2=1
\end{array}
\right.
\]
\[
\Leftrightarrow
\left\lbrace
\begin{array}{c c}
A = -\frac{1}{b^2}\\
B = \frac{1}{2b^2}\\
C = \frac{1}{2b^2}
\end{array}
\right.
\]
\[
\Leftrightarrow
\frac{1}{x(x^2-b^2)} = -\frac{1}{b^2x^2} + \frac{1}{2b^2(x+b)} + \frac{1}{2b^2(x-b)}
\]

\subsection*{1.6}
\subsubsection*{a)}
\[
\begin{pmatrix}
  1 & 2 & -3 & 6 \\
  2 & -1 & 4 & 2 \\
  4 & 3 & -2 & a \\
 \end{pmatrix}
  \overset{\begin{matrix}
  R2 = R2 - 2R1 \\
  R3 = R3 - 4R1
 \end{matrix}}{\rightarrow}
\begin{pmatrix}
  1 & 2 & -3 & 6 \\
  0 & -5 & 10 & -10 \\
  0 & -5 & 10 & (a-24) \\
 \end{pmatrix}
 \overset{\begin{matrix}
  R2 = -\frac{1}{5}R2 \\
  R3 = -\frac{1}{5}R3
 \end{matrix}}{\rightarrow}
 \begin{pmatrix}
  1 & 2 & -3 & 6 \\
  0 & 1 & -2 & 2 \\
  0 & 1 & -2 & -\frac{(a-24)}{5} \\
 \end{pmatrix}
\]
Antwoord:\\
Voor $a\neq14$ zal het stelsel geen oplossingen hebben.
Voor $a=14$ zal het stelsel oneindig veel oplossingen hebben.

(voor $a=14$)
\[
\begin{pmatrix}
  1 & 2 & -3 & 6 \\
  0 & 1 & -2 & 2 \\
  0 & 0 & 0 & 0 \\
 \end{pmatrix}
\]

Wat we nog kunnen rijreduceren tot:
\[
\begin{pmatrix}
  1 & 0 & 1 & 2 \\
  0 & 1 & -2 & 2 \\
  0 & 0 & 0 & 0 \\
 \end{pmatrix}
\]

We constateren dat $z$ hier een vrije variabele is en wijzen haar de waarde $\lambda$ toe. Dan is onze oplossingsverzameling de volgende:\\
$ V = \left\{ \left( 2 - \lambda, 2 + 2\lambda, \lambda \right) | \lambda \in \mathbb{R} \right\} $

\subsubsection*{b)}
\[
 \left(
\begin{array}{cc|c}
1 & a & (a+1) \\
a & 1 & 2 \\
\end{array}
\right)
 \overset{\begin{matrix}
  R2 = R2-aR1
 \end{matrix}}{\rightarrow}
 \left(
\begin{array}{cc|c}
1 & a & (a+1) \\
0 & (1-a^2) & (a^2-a+2) \\
\end{array}
\right)
\]
Antwoord:\\
Voor $a=1$: oneindig veel oplossingen,
$\left\lbrace
\begin{array}{l l}
x = 2 − \lambda \\
y = \lambda
\end{array}
\right.$\\
Voor $a=-1$: geen oplossingen.\\
Voor $a \neq 1 \wedge a \neq -1$: precies 1 oplossing,
$\left\lbrace
\begin{array}{l l}
x = \frac{1}{a+1} \\
y = \frac{a+2}{a+1}
\end{array}
\right.$

\subsubsection*{c)}
\[
 \left(
\begin{array}{ccc|c}
a & (a+1) & 1 & 0 \\
a & 1 & (a+1) & 0 \\
2a & 1 & 1 & (a+1)
\end{array}
\right)
\overset{\begin{matrix}
  R2 = R2-R1 \\
  R3 = R3-2R1
 \end{matrix}}{\rightarrow}
 \left(
\begin{array}{ccc|c}
a & (a+1) & 1 & 0 \\
0 & -a & a & 0 \\
0 & -(2a+1) & -1 & (a+1)
\end{array}
\right)
\]
\[
\overset{\begin{matrix}
  R2 = -\frac{1}{a}R2
 \end{matrix}}{\rightarrow}
 \left(
\begin{array}{ccc|c}
a & (a+1) & 1 & 0 \\
0 & 1 & -1 & 0 \\
0 & -(2a+1) & -1 & (a+1)
\end{array}
\right)
\overset{\begin{matrix}
	R1=R1-(a+1)R2\\
	R3=R3+(2a+1)R2
 \end{matrix}}{\rightarrow}
 \left(
\begin{array}{ccc|c}
a & 0 & (a+2) & 0 \\
0 & 1 & -1 & 0 \\
0 & 0 & -2(a+1) & (a+1)
\end{array}
\right)
\]
\[
\overset{\begin{matrix}
	R3=-\frac{2}{(a+1)}R3
 \end{matrix}}{\rightarrow}
 \left(
\begin{array}{ccc|c}
a & 0 & (a+2) & 0 \\
0 & 1 & -1 & 0 \\
0 & 0 & 1 & -\frac{1}{2}
\end{array}
\right)
\overset{\begin{matrix}
	R1=R1-(a+2)R3\\
	R2=R2+R3
 \end{matrix}}{\rightarrow}
 \left(
\begin{array}{ccc|c}
a & 0 & 0 & \frac{a+2}{2} \\
0 & 1 & 0 & -\frac{1}{2} \\
0 & 0 & 1 & -\frac{1}{2}
\end{array}
\right)
\]
\[
\overset{\begin{matrix}
	R1=\frac{1}{a}R1
 \end{matrix}}{\rightarrow}
 \left(
\begin{array}{ccc|c}
1 & 0 & 0 & \frac{a+2}{2a} \\
0 & 1 & 0 & -\frac{1}{2} \\
0 & 0 & 1 & -\frac{1}{2}
\end{array}
\right)
\]
Antwoord:\\
Voor $a=0$: geen oplossingen.\\
Voor $a=-1$: oneindig veel oplossingen:
$
\left\{ 
  \begin{array}{l l}
    x=\lambda\\
	y=\lambda\\
	z=\lambda
  \end{array} \right.
$\\
Voor $a\neq-1\wedge a\neq0$: precies 1 oplossing:
$
\left\{ 
  \begin{array}{l l}
    x=\frac{a+2}{2a}\\
	y=-\frac{1}{2}\\
	z=-\frac{1}{2}
  \end{array} \right.
$

\subsection*{1.8}
\[
\begin{pmatrix}
  k & 5 & 3 \\
  5 & 1 & -1 \\
  k & 2 & 1 \\
 \end{pmatrix}
 \overset{\begin{matrix}
  R1 = R1-\frac{k}{5}R2 \\
  R3 = R3-\frac{k}{5}R2
 \end{matrix}}{\rightarrow}
 \begin{pmatrix}
  0 & 5-\frac{k}{5} & 3+\frac{k}{5} \\
  5 & 1 & -1 \\
  0 & 2-\frac{k}{5} & 1+\frac{k}{5} \\
 \end{pmatrix}
\]
Antwoord:\\
Enkel als $k=1$ zullen R1 en R3 lineair afhankelijk zijn en zullen er niet-triviale oplossingen zijn, anders is er enkel de triviale nuloplossing.

\subsection*{1.10}
\subsubsection*{a)}
\[
\left(
\begin{array}{ccc|c}
3 & 1 & 1 & 0 \\b
1 & 3 & 1 & 0 \\
1 & 1 & 3 & 0
\end{array}
\right)
 \overset{\begin{matrix}
  R1 = R1-3R2 \\
  R3 = R3-R2
 \end{matrix}}{\rightarrow}
 \left(
\begin{array}{ccc|c}
0 & -8 & -2 & 0 \\
1 & 3 & 1 & 0 \\
0 & -2 & 2 & 0
\end{array}
\right)
 \overset{\begin{matrix}
  R1 = R2 \\
  R2 = -\frac{1}{2}R3 \\
  R3 = -\frac{1}{2}R1
 \end{matrix}}{\rightarrow}
 \left(
 \begin{array}{ccc|c}
1 & 3 & 1 & 0 \\
0 & 1 & -1 & 0 \\
0 & 4 & 1 & 0
\end{array}
\right)
\]
\[
\overset{\begin{matrix}
  R1 = R1-3R2 \\
  R3 = R3-4R2
 \end{matrix}}{\rightarrow}
 \left(
 \begin{array}{ccc|c}
1 & 0 & 4 & 0 \\
0 & 1 & -1 & 0 \\
0 & 0 & 5 & 0
\end{array}
\right)
 \overset{\begin{matrix}
  R3 = \frac{1}{5}R3
 \end{matrix}}{\rightarrow}
 \left(
 \begin{array}{ccc|c}
1 & 0 & 4 & 0 \\
0 & 1 & -1 & 0 \\
0 & 0 & 1 & 0
\end{array}
\right)
\overset{\begin{matrix}
  R1 = R1-4R3\\
  R2=R2+R3
 \end{matrix}}{\rightarrow}
 \left(
 \begin{array}{ccc|c}
1 & 0 & 0 & 0 \\
0 & 1 & 0 & 0 \\
0 & 0 & 1 & 0
\end{array}
\right)
\]
Antwoord:\\
Enkel de tiriviale oplossing $\vec{x} = \vec{0}$, want $Rang(A) = n$

\subsubsection*{c)}
\[
\begin{pmatrix}
  1 & 2 & -1 \\
  2 & 1 & 1 \\
  2 & -1 & 7 \\
  5 & 5 & 4 \\
  9 & 7 & 6 
 \end{pmatrix}
 \overset{\begin{matrix}
  R2 = R2-2R1 \\
  R3 = R3-2R1 \\
  R4 = R4-5R1 \\
  R5 = R5-9R1
 \end{matrix}}{\rightarrow}
 \begin{pmatrix}
  1 & 2 & -1 \\
  0 & -3 & 3 \\
  0 & -5 & 5 \\
  0 & -5 & 9 \\
  0 & -11 & 15 
 \end{pmatrix}
 \overset{\begin{matrix}
  R2 = -\frac{1}{3}R2\\
  R2\:en\:R3\:zijn\:lineair\\afhankelijk,\;schrap\:R3
 \end{matrix}}{\rightarrow}
  \begin{pmatrix}
  1 & 2 & -1 \\
  0 & 1 & -1 \\
  0 & -5 & 9 \\
  0 & -11 & 15 
 \end{pmatrix}
\]
\[
\overset{\begin{matrix}
 R1 = R1-2R2\\
 R3 = R3+5R2\\
 R4 = R4+11R2
 \end{matrix}}{\rightarrow}
 \begin{pmatrix}
  1 & 0 & 1 \\
  0 & 1 & -1 \\
  0 & 0 & 4 \\
  0 & 0 & 4 
 \end{pmatrix}
 \overset{\begin{matrix}
 R3 = \frac{1}{4}R3\\
 R3\:en\:R4\:zijn\:lineair\\afhankelijk,\;schrap\:R4
 \end{matrix}}{\rightarrow}
 \begin{pmatrix}
  1 & 0 & 1 \\
  0 & 1 & -1 \\
  0 & 0 & 1 \\
 \end{pmatrix}
 \overset{\begin{matrix}
 R1 = R1-R3\\
 R2 = R2+R3\\
 \end{matrix}}{\rightarrow}
 \begin{pmatrix}
  1 & 0 & 0 \\
  0 & 1 & 0 \\
  0 & 0 & 1 \\
 \end{pmatrix}
\]
Antwoord:\\
\[
\left\lbrace
\begin{array}{l l}
x=0\\
y=0\\
z=0
\end{array}
\right.
\]

\subsection*{1.12}
\subsubsection*{a)}
\[
 \begin{pmatrix}
1 & 5 & 2 & 3 \\
4 & 2 & -1 & -6 \\
-5 & 1 & 3 & 11
 \end{pmatrix}
  \overset{\begin{matrix}
  R2 = R2-4R1 \\
  R3 = R3+5R1
 \end{matrix}}{\rightarrow}
 \begin{pmatrix}
1 & 5 & 2 & 3 \\
0 & -18 & -9 & -18 \\
0 & 26 & 13 & 26
 \end{pmatrix}
 \overset{\begin{matrix}
  R2 = -\frac{1}{9} \\
  R2\:en\:R3\:zijn\:lineair\\afhankelijk,\;schrap\:R3
 \end{matrix}}{\rightarrow}
 \begin{pmatrix}
1 & 5 & 2 & 3 \\
0 & 2 & 1 & 2 \\
0 & 0 & 0 & 0
 \end{pmatrix}
\]
Antwoord:\\
Lineair afhankelijk. Een zo groot mogelijke lineair onafhankelijke deelverzamelijk wordt bv. gegeven door de eerste 2 vectoren.\\
$\lambda_3 = p met p$ willekeurig en $\lambda_4 = t$ met t willekeurig.\\
$2\lambda_2 = -p-2t \Rightarrow \lambda_2=\frac{1}{2}p-t$\\
$\lambda_1=-5(-\frac{1}{2}p-t) -2p-3t=\frac{5}{2}p+5t-2p = \frac{1}{2}p-2t$
\[
(\frac{1}{2}-2t)\begin{pmatrix}1\\4\\-5\end{pmatrix}+(\frac{1}{2}p-t)\begin{pmatrix}5\\2\\1\end{pmatrix}+p\begin{pmatrix}2\\-1\\3\end{pmatrix}+t
\begin{pmatrix}3\\-6\\11\end{pmatrix} = \vec{0}
\]

\subsection*{1.13}
\subsubsection*{a)}
\[
\begin{pmatrix}
  1 & -4 & 2 \\
  -4 & 1 & -2 \\
  2 & -2 & -2 
 \end{pmatrix}
 \overset{\begin{matrix}
  R2 = R2+4R1 \\
  R3 = R3-2R1
 \end{matrix}}{\rightarrow}
 \begin{pmatrix}
  1 & -4 & 2 \\
  0 & -15 & 6 \\
  0 & 6 & -6 
 \end{pmatrix}
 \overset{\begin{matrix}
  R3 = \frac{6}{15}R2
 \end{matrix}}{\rightarrow}
 \begin{pmatrix}
  1 & -4 & 2 \\
  0 & -15 & 6 \\
  0 & 0 & -\frac{36}{10} 
 \end{pmatrix}
\]
Antwoord:\\
Deze drie vectoren zijn lineair onafhankelijk.

\subsubsection*{b)}
\[
 \begin{pmatrix}
  1 & 1 & -1 \\
  2 & -1 & -5 \\
  -5 & 4 & 14 \\
  3 & -1 & -17 
 \end{pmatrix}
 \overset{\begin{matrix}
  R2 = R2-2R1\\
  R3 = R3+5R1\\
  R4 = R4-3R1
 \end{matrix}}{\rightarrow}
 \begin{pmatrix}
  1 & 1 & -1 \\
  0 & -3 & -3 \\
  0 & 9 & 9 \\
  0 & -4 & 14 
 \end{pmatrix}
\]
Antwoord:\\
Lineair afhankelijk, een zo groot mogelijk lineair onafhakelijke deelverzameling wordt bv. gegeven door de eerste 3 vectoren.

\subsubsection*{c)}
\[
\begin{pmatrix}
  1 & -1 & 1 & 2 \\
  -2 & 5 & 0 & -15 \\
  3 & 3 & 2 & 1
 \end{pmatrix}
 \overset{\begin{matrix}
  R2 = R2+2R1\\
  R3 = R3-3R1
 \end{matrix}}{\rightarrow}
 \begin{pmatrix}
  1 & -1 & 1 & 2 \\
  0 & 3 & 2 & -11 \\
  0 & 6 & -1 & -5
 \end{pmatrix}
 \overset{\begin{matrix}
  R3 = R3-2R2
 \end{matrix}}{\rightarrow}
 \begin{pmatrix}
  1 & -1 & 1 & 2 \\
  0 & 3 & 2 & -11 \\
  0 & 0 & -5 & 17
 \end{pmatrix}
\]
Antwoord:\\
Lineair onafhankelijk?

\subsection*{1.14}
\[
 \begin{pmatrix}
  cos(\theta) & 1\\
  1 & 2cos(\theta)
 \end{pmatrix}
 \overset{\begin{matrix}
  R1 = R1-cos(\theta)R2
 \end{matrix}}{\rightarrow}
 \begin{pmatrix}
  0 & 1-2cos^2(\theta)\\
  1 & 2cos(\theta)
 \end{pmatrix}
\]
Lineair afhankelijk asa
\begin{align*}
1-2cos^2(\theta) &= 0\\
cos^2(\theta) &= \frac{1}{2}\\
cos(\theta) &= \pm\frac{\sqrt{2}}{2}\\
\theta &= \pm\frac{\pi}{4} \vee \theta = \pm\frac{3\pi}{4}
\end{align*}


\section*{Hoofdstuk 2}
\subsection*{2.1}
\subsubsection*{a)}
\[
A+B=
\begin{pmatrix}
  1 & -2 & 3 \\
  0 & 3 & 4 \\
 \end{pmatrix}
 +
\begin{pmatrix}
  0 & 1 & -4 \\
  2 & -3 & 0 \\
 \end{pmatrix}
 =
 \begin{pmatrix}
  1 & -1 & -1 \\
  2 & 0 & 4 \\
 \end{pmatrix}
\]
\subsubsection*{f)}
\[
2P-Q=2
\begin{pmatrix}
  1 & -2 \\
  0 & 4 \\
 \end{pmatrix}
 -
\begin{pmatrix}
  3 & 0 \\
  0 & 1 \\
 \end{pmatrix}
 =
 \begin{pmatrix}
  -1 & -4 \\
  0 & 7 \\
 \end{pmatrix}
\]
\subsubsection*{k)}
\[
P.C=
\begin{pmatrix}
  1 & -2 \\
  0 & 4 \\
 \end{pmatrix}
 .
\begin{pmatrix}
  -5 & 3 \\
  4 & -1 \\
  2 & -1
 \end{pmatrix}
 Onmogelijk
\]
\subsubsection*{p)}
\[
\vec{b}.C=
\begin{pmatrix}
  2 & 5 & -2 \\
 \end{pmatrix}
 .
\begin{pmatrix}
  -5 & 3 \\
  4 & -1 \\
  2 & -1
 \end{pmatrix}
 =
 \begin{pmatrix}
  (-10+20-4) & (6-5+2) \\
 \end{pmatrix}
 =
 \begin{pmatrix}
  6 & 3 \\
 \end{pmatrix}
\]

\subsection*{2.3}
\subsubsection*{a)}
\[
A.B=
 \begin{pmatrix}
  2 & -1 \\
  -1 & 1
 \end{pmatrix}
 .
 \begin{pmatrix}
  -1 & 3 \\
  3 & 2
 \end{pmatrix}
 =
 \begin{pmatrix}
  -5 & 4 \\
  4 & -1
 \end{pmatrix}
 = A.B\;(toevallig)
\]
\subsubsection*{b)}
\[
A.B=
\begin{pmatrix}
1 & 0 \\
1 & 0
\end{pmatrix}
.
\begin{pmatrix}
0 & 1 \\
-1 & 0
\end{pmatrix}
=
\begin{pmatrix}
0 & 1 \\
0 & 1
\end{pmatrix}
\]
\[
A.B=
\begin{pmatrix}
0 & 1 \\
-1 & 0
\end{pmatrix}
.
\begin{pmatrix}
1 & 0 \\
1 & 0
\end{pmatrix}
=
\begin{pmatrix}
1 & 0 \\
-1 & 0
\end{pmatrix}
\]
\subsection*{2.4}
\[
\begin{pmatrix}
a & b \\
c & d
\end{pmatrix}
.
\begin{pmatrix}
2 & 1 \\
3 & 2
\end{pmatrix}
=
\begin{pmatrix}
3 & 2 \\
1 & 4
\end{pmatrix}
\rightarrow
\left(
\begin{array}{cccc|c}
2 & 1 & 0 & 0 & 3\\
3 & 2 & 0 & 0 & 2\\
0 & 0 & 2 & 3 & 1\\
0 & 0 & 1 & 2 & 4
\end{array}
\right)
\overset{GRM}{\rightarrow}
\left\lbrace
\begin{array}{l l}
a=0\\
b=1\\
c=-10\\
d=7
\end{array}
\right.
\]

\subsection*{2.5}
\subsubsection*{a)}
\[
\begin{pmatrix}
2 & -1 \\
-1 & 1
\end{pmatrix}
.
\begin{pmatrix}
-1 & 3 \\
3 & 2
\end{pmatrix}
-
\begin{pmatrix}
-1 & 3 \\
3 & 2
\end{pmatrix}
.
\begin{pmatrix}
2 & -1 \\
-1 & 1
\end{pmatrix}
\]
\[
=
\begin{pmatrix}
-5 & 4 \\
4 & -1
\end{pmatrix}
-
\begin{pmatrix}
-5 & 4 \\
4 & -1
\end{pmatrix}
=
\begin{pmatrix}
0 & 0 \\
0 & 0
\end{pmatrix}
\]

\subsubsection*{b)}
\[
\begin{pmatrix}
1 & 0 \\
1 & 0
\end{pmatrix}
.
\begin{pmatrix}
0 & 1\\
-1 & 0
\end{pmatrix}
-
\begin{pmatrix}
0 & 1\\
-1 & 0
\end{pmatrix}
.
\begin{pmatrix}
1 & 0 \\
1 & 0
\end{pmatrix}
\]
\[
=
\begin{pmatrix}
0 & 1\\
0 & 1
\end{pmatrix}
-
\begin{pmatrix}
1 & 0 \\
-1 & 0
\end{pmatrix}
=
\begin{pmatrix}
-1 & 1 \\
1 & 1
\end{pmatrix}
\]

\subsection*{2.6}
\[
\begin{pmatrix}
0 & 1 \\
1 & 0
\end{pmatrix}^2
+
\begin{pmatrix}
0 & -i \\
i & 0
\end{pmatrix}^2
+
\begin{pmatrix}
1 & 0 \\
0 & -1
\end{pmatrix}^2
\]
\[
=
\begin{pmatrix}
1 & 0 \\
0 & 1
\end{pmatrix}
+
\begin{pmatrix}
1 & 0 \\
0 & 1
\end{pmatrix}
+
\begin{pmatrix}
1 & 0 \\
0 & 1
\end{pmatrix}
=
\begin{pmatrix}
3 & 0 \\
0 & 3
\end{pmatrix}
\]

\subsection*{2.7}
\subsubsection*{b)}
\[
\begin{pmatrix}
1 & 2 \\
2 & 0
\end{pmatrix}
.
\begin{pmatrix}
a & b \\
c & d
\end{pmatrix}
=
\begin{pmatrix}
a & b \\
c & d
\end{pmatrix}
.
\begin{pmatrix}
1 & 2 \\
2 & 0
\end{pmatrix}
\Leftrightarrow
\left\lbrace
\begin{array}{l l}
2c=2b\\
b+2d=2a\\
2a=c+2d\\
2b=2c
\end{array}
\right.
\leftrightarrow
\left\lbrace
\begin{array}{l l}
a=d\\
b=2a-2d\\
c=2d-2a\\
d=a
\end{array}
\right.
\]
\subsection*{2.8}
\subsubsection*{a)}
Neen, want $2AB$ komt van $AB+BA$ en $AB \neq BA$.
\subsubsection*{b)}
Ja.

\subsection*{2.9}
\subsubsection*{a)}
\[
\left(
\begin{array}{cc|cc}
2 & -3 & 1 & 0 \\
4 & 1 & 0 & 1
\end{array}
\right)
\overset{\begin{matrix}
  R2 = R2-2R1
 \end{matrix}}{\rightarrow}
 \left(
\begin{array}{cc|cc}
2 & -3 & 1 & 0 \\
0 & 7 & -2 & 1
\end{array}
\right)
\overset{\begin{matrix}
  R1 = R1+\frac{3}{7}R2
 \end{matrix}}{\rightarrow}
  \left(
\begin{array}{cc|cc}
2 & 0 & \frac{1}{7} & \frac{3}{7} \\
0 & 7 & -2 & 1
\end{array}
\right)
\]
\[
\overset{\begin{matrix}
  R1 = \frac{1}{2}R1\\
  R2 = \frac{1}{7}R2
 \end{matrix}}{\rightarrow}
   \left(
\begin{array}{cc|cc}
1 & 0 & \frac{1}{14} & \frac{3}{14} \\
0 & 1 & -\frac{2}{7} & \frac{1}{7}
\end{array}
\right)
\\\;\;\; Antwoord: \;
\begin{pmatrix}
\frac{1}{14} & \frac{3}{14} \\
-\frac{2}{7} & \frac{1}{7}
\end{pmatrix}
\]

\subsubsection*{f)}
\[
\left(
\begin{array}{cccc|cccc}
3 & 4 & 0 & 0 & 1 & 0 & 0 & 0\\
1 & 2 & 0 & 0 & 0 & 1 & 0 & 0\\
0 & 0 & 3 & 1 & 0 & 0 & 1 & 0\\
0 & 0 & 4 & 2 & 0 & 0 & 0 & 1
\end{array}
\right)
\rightarrow
\left(
\begin{array}{cccc|cccc}
1 & 2 & 0 & 0 & 0 & 1 & 0 & 0\\
3 & 4 & 0 & 0 & 1 & 0 & 0 & 0\\
0 & 0 & 3 & 1 & 0 & 0 & 1 & 0\\
0 & 0 & 4 & 2 & 0 & 0 & 0 & 1
\end{array}
\right)
\overset{\begin{matrix}
  R2 = R2-3R1
 \end{matrix}}{\rightarrow}
\]
\[
\left(
\begin{array}{cccc|cccc}
1 & 2 & 0 & 0 & 0 & 1 & 0 & 0\\
0 & -2 & 0 & 0 & 1 & -3 & 0 & 0\\
0 & 0 & 3 & 1 & 0 & 0 & 1 & 0\\
0 & 0 & 4 & 2 & 0 & 0 & 0 & 1
\end{array}
\right)
\overset{\begin{matrix}
R1 = R1+R2
\end{matrix}}{\rightarrow}
\left(
\begin{array}{cccc|cccc}
1 & 0 & 0 & 0 & 1 & -2 & 0 & 0\\
0 & -2 & 0 & 0 & 1 & -3 & 0 & 0\\
0 & 0 & 3 & 1 & 0 & 0 & 1 & 0\\
0 & 0 & 4 & 2 & 0 & 0 & 0 & 1
\end{array}
\right)
\overset{\begin{matrix}
R4=R4-\frac{4}{3}R3
\end{matrix}}{\rightarrow}
\]
\[
\left(
\begin{array}{cccc|cccc}
1 & 0 & 0 & 0 & 1 & -2 & 0 & 0\\
0 & -2 & 0 & 0 & 1 & -3 & 0 & 0\\
0 & 0 & 3 & 1 & 0 & 0 & 1 & 0\\
0 & 0 & 0 & \frac{2}{3} & 0 & 0 & -\frac{4}{3} & 1
\end{array}
\right)
\overset{\begin{matrix}
R3=R3-\frac{3}{2}R4
\end{matrix}}{\rightarrow}
\left(
\begin{array}{cccc|cccc}
1 & 0 & 0 & 0 & 1 & -2 & 0 & 0\\
0 & -2 & 0 & 0 & 1 & -3 & 0 & 0\\
0 & 0 & 3 & 0 & 0 & 0 & 3 & -\frac{3}{2}\\
0 & 0 & 0 & \frac{2}{3} & 0 & 0 & -\frac{4}{3} & 1
\end{array}
\right)
\overset{\begin{matrix}
R2 = -\frac{1}{2}R2\\
R3 = \frac{1}{3}R3\\
R4 = \frac{3}{2}R4
\end{matrix}}{\rightarrow}
\]
\[
\left(
\begin{array}{cccc|cccc}
1 & 0 & 0 & 0 & 1 & -2 & 0 & 0\\
0 & 1 & 0 & 0 & -\frac{1}{2} & \frac{3}{2} & 0 & 0\\
0 & 0 & 1 & 0 & 0 & 0 & 1 & -\frac{1}{2}\\
0 & 0 & 0 & 1 & 0 & 0 & -2 & \frac{3}{2}
\end{array}
\right)
\]
Antwoord:\\
\[
\begin{pmatrix}
1 & -2 & 0 & 0\\
-\frac{1}{2} & \frac{3}{2} & 0 & 0\\
0 & 0 & 1 & -\frac{1}{2}\\
0 & 0 & -2 & \frac{3}{2}
\end{pmatrix}
\]

\subsection*{2.10}
\[
\left(
\begin{array}{ccc|ccc}
2 & 5 & 1 & 1 & 0 & 0\\
3 & 1 & 2 & 0 & 1 & 0\\
-2 & 1 & 0 & 0 & 0 & 1\\
\end{array}
\right)
\overset{GRM}{\rightarrow}
\left(
\begin{array}{ccc|ccc}
19 & 0 & 0 & 2 & -1 & -9\\
0 & 19 & 0 & 5 & -2 & 1\\
0 & 0 & 19 & -5 & 12 & 13\\
\end{array}
\right)
\]
Antwoord:\\
\[
\frac{1}{19}
\begin{pmatrix}
2 & -1 & -9\\
4 & -2 & 1\\
-5 & 12 & 13
\end{pmatrix}
\]

\subsection*{2.14}
\subsubsection*{a)}
\[
\begin{pmatrix}
cos(\theta_1) & -sin(\theta_1)\\
sin(\theta_1) & cos(\theta_1)
\end{pmatrix}
.
\begin{pmatrix}
cos(\theta_2) & -sin(\theta_2)\\
sin(\theta_2) & cos(\theta_2)
\end{pmatrix}
\]
\[
=
\begin{pmatrix}
cos(\theta_1)cos(\theta_2) - sin(\theta_1)sin(\theta_2) & -(cos(\theta_1)sin(\theta_2)+cos(\theta_2)sin(\theta_1))\\
sin(\theta_1)cos(\theta_2)+cos(\theta_1)sin(\theta_2) &
cos(\theta_1)cos(\theta_2) - -sin(\theta_1)sin(\theta_2)
\end{pmatrix}
\]
\[
=
\begin{pmatrix}
cos(\theta_1+\theta_2) & -sin(\theta_1+\theta_2)\\
sin(\theta_1+\theta_2) & cos(\theta_1+\theta_2)
\end{pmatrix}
\]

\subsection*{2.17}
\subsubsection*{a)}
\[
\begin{pmatrix}
1 & 1\\
0 & 0
\end{pmatrix}
\;\;\; want
\begin{pmatrix}
1 & 1\\
0 & 0
\end{pmatrix}
.
\begin{pmatrix}
1 & 1\\
0 & 0
\end{pmatrix}
=
\begin{pmatrix}
(1*1-1*0) & (1*1-1*0)\\
(0*1-0*0) & (0*1-0*0)
\end{pmatrix}
=
\begin{pmatrix}
1 & 1\\
0 & 0
\end{pmatrix}
\]

\section*{Hoofdstuk 3}
\subsection*{3.1}
\[
\begin{vmatrix}
4 & 1\\
3 & 2
\end{vmatrix}
= 5\;
D1 =
\begin{vmatrix}
11 & 1\\
12 & 2
\end{vmatrix}
= 10\;
D2 =
\begin{vmatrix}
4 & 11\\
3 & 22
\end{vmatrix}
=15
\]
\[
x = 2 \;en\; y = 3
\]
\subsection*{3.2}
\subsubsection*{a)}
\[
\begin{vmatrix}
2 & 0 \\
0 & 3
\end{vmatrix}
=6
\]

\subsubsection*{b)}
\[
\begin{vmatrix}
0 & 1\\
-2 & 3
\end{vmatrix}
=2
\]

\subsubsection*{c)}
\[
\begin{vmatrix}
cos(n\theta) & -sin(n\theta) \\
sin(n\theta) & cos(n\theta)
\end{vmatrix}
=cos^2(n\theta) + sin^2(n\theta) =1
\]

\subsection*{3.3}
\[D=
\begin{vmatrix}
1 & 1 & 1 \\
1 & 2 & 3 \\
1 & 4 & 9 
\end{vmatrix}
=2\]
\[D_1=
\begin{vmatrix}
6 & 1 & 1 \\
14 & 2 & 3 \\
36 & 4 & 9
\end{vmatrix}
=2,\;D_2=
\begin{vmatrix}
1 & 6 & 1 \\
1 & 14 & 3 \\
1 & 36 & 9
\end{vmatrix}
=4,\;D_3=
\begin{vmatrix}
1 & 1 & 6 \\
1 & 2 & 14 \\
1 & 4 & 36
\end{vmatrix}
=6
\]
$$x_1=\frac{D_1}{D},\;x_2=\frac{D_2}{D},\;x_3=\frac{D_3}{D}$$
Antwoord:\\
\[
\left\lbrace
\begin{array}{l l}
x_1=1\\
x_2=2\\
x_3=3
\end{array}
\right.
\]

\subsection*{3.4}
\subsubsection*{a)}
\[
\begin{vmatrix}
2 &-1 & 3\\
4 & 1 & -2\\
-3 & 2 & 1
\end{vmatrix}
=2.
\begin{vmatrix}
1 & -2 \\
2 & 1
\end{vmatrix}
-(-1).
\begin{vmatrix}
4 & -2 \\
-3 & 1
\end{vmatrix}
+3.
\begin{vmatrix}
4 & 1 \\
-3 & 2
\end{vmatrix}
= 10-2+33=41
\]
\[
\begin{vmatrix}
2 &-1 & 3\\
4 & 1 & -2\\
-3 & 2 & 1
\end{vmatrix}
=-(-1).
\begin{vmatrix}
4 & -2 \\
3 & 1
\end{vmatrix}
+1.
\begin{vmatrix}
2 & 3 \\
-3 & 1
\end{vmatrix}
-2
\begin{vmatrix}
2 & 3 \\
4 & -2
\end{vmatrix}
=-2+11+32=41
\]

\subsubsection*{c)}
\[
\begin{vmatrix}
a & b & c \\
c & a & b \\
b & c & a
\end{vmatrix}
=
a.\begin{vmatrix}
a & b \\
c & a \\
\end{vmatrix}
-c.\begin{vmatrix}
b & c \\
c & a \\
\end{vmatrix}
+b.\begin{vmatrix}
b & c \\
a & b \\
\end{vmatrix}
=
a.(a^2-bc) - c.(ab-c^2) + b.(b^2-ac) = a^3+b^3+c^3 +3abc
\]
\[
\begin{vmatrix}
a & b & c \\
c & a & b \\
b & c & a
\end{vmatrix}
=
-b.\begin{vmatrix}
c & b \\
b & a \\
\end{vmatrix}
+a.\begin{vmatrix}
a & c \\
b & a \\
\end{vmatrix}
-c.\begin{vmatrix}
a & c \\
c & b \\
\end{vmatrix}
=
-b.(ac-b^2) + a.(a^2-bc) - c.(ab-c^2) = a^3+b^3+c^3 +3abc
\]

\subsubsection*{e)}
\[
\begin{vmatrix}
0 & 3 & 2 \\
2 & 0 & 1 \\
2 & 6 & 0
\end{vmatrix}
=0.\begin{vmatrix}
0 & 1 \\
6 & 0 \\
\end{vmatrix}
-3.\begin{vmatrix}
2 & 1 \\
2 & 0
\end{vmatrix}
+2.\begin{vmatrix}
2 & 0 \\
2 & 6
\end{vmatrix}
=18
\]
\[
\begin{vmatrix}
0 & 3 & 2 \\
2 & 0 & 1 \\
2 & 6 & 0
\end{vmatrix}
=-3.\begin{vmatrix}
2 & 1 \\
2 & 0 \\
\end{vmatrix}
+0.\begin{vmatrix}
0 & 2 \\
2 & 0
\end{vmatrix}
-6.\begin{vmatrix}
0 & 2 \\
2 & 1
\end{vmatrix}
=18
\]

\subsection*{3.6}
\subsubsection*{b)}
\[
\begin{vmatrix}
3 & 4 & 0 & 0\\
1 & 2 & 0 & 0\\
0 & 0 & 3 & 1\\
0 & 0 & 4 & 2
\end{vmatrix}
=3.2.2-1.4.2 = 4
\]

\subsubsection*{c)}
\[
\begin{vmatrix}
-2 & 6 & 17 & -5\\
0 & 3 & 22 & -12\\
0 & 0 & 4 & -12\\
0 & 0 & 0 & -6
\end{vmatrix}
=(-2).3.4.(-6)=144
\]

\subsubsection*{d)}
\[
\begin{vmatrix}
0 & 2 & 0 & 0 \\
2 & 0 & 2 & 0 \\
0 & 2 & 0 & 2 \\
0 & 0 & 2 & 0
\end{vmatrix}
=
-2\begin{vmatrix}
2 & 2 & 0\\
0 & 0 & 2\\
0 & 2 & 0
\end{vmatrix}
=-2.2\begin{vmatrix}
0 & 2 \\
2 & 0
\end{vmatrix}
=16
\]

\subsection*{3.8}
\subsubsection*{b)}
\[
\begin{vmatrix}
1 & 2x & 3x^2\\
2x^3 & 3x^4 & 4x^5\\
3x^6 & 4x^7 & 6x^8
\end{vmatrix}
\overset{\begin{matrix}
R2 = R2-2x^3R1\\
R3 = R3-3x^6R1
\end{matrix}}{=}
\begin{vmatrix}
1 & 2x & 3x^2\\
0 & -x^4 & -x^5\\
0 & -2x^7 & -3x^8
\end{vmatrix}
=0\;\Leftrightarrow\;
\begin{vmatrix}
-x^4 & -x^5\\
-2x^7 & -3x^8
\end{vmatrix}=0
\]
\[\Leftrightarrow 3x^12 - 2x^12 = x ^12 = 0\;
\Leftrightarrow\;x=0
\]

\subsection*{3.9}
\subsubsection*{a)}
\[
\begin{vmatrix}
3 & 2 & -2 \\
6 & 1 & 5 \\
-9 & 3 & 4
\end{vmatrix}
\overset{\begin{matrix}
R2 = R2-2R\\
R3=R3+3R1
\end{matrix}}{=}
\begin{vmatrix}
3 & 2 & -2 \\
0 & -3 & 9 \\
0 & 9 & -2
\end{vmatrix}
\overset{\begin{matrix}
R3=-\frac{1}{3}R3
\end{matrix}}{=}
-3.
\begin{vmatrix}
3 & 2 & -2 \\
0 & 1 & -3 \\
0 & 9 & -2
\end{vmatrix}
\overset{\begin{matrix}
R3=R3-9R2
\end{matrix}}{=}
-3.
\begin{vmatrix}
3 & 2 & -2 \\
0 & 1 & -3 \\
0 & 0 & 25
\end{vmatrix}
\]
\[
= (-3)*3*25=-225
\]

\subsubsection*{b)}
\[
\begin{vmatrix}
1 & 0 & 1 & 1 \\
0 & 1 & 1 & 0 \\
1 & 0 & 0 & 1 \\
1 & 1 & 1 & 0
\end{vmatrix}
\overset{\begin{matrix}
R3=R3-R1\\
R4=R4-R1
\end{matrix}}{=}
\begin{vmatrix}
1 & 0 & 1 & 1 \\
0 & 1 & 1 & 0 \\
0 & 0 & -1 & 0 \\
0 & 1 & 0 & -1
\end{vmatrix}
\overset{\begin{matrix}
R4=R4-R2
\end{matrix}}{=}
\]\[
\begin{vmatrix}
1 & 0 & 1 & 1 \\
0 & 1 & 1 & 0 \\
0 & 0 & -1 & 0 \\
0 & 0 & -1 & -1
\end{vmatrix}
\overset{\begin{matrix}
R4=R4-R3
\end{matrix}}{=}
\begin{vmatrix}
1 & 0 & 1 & 1 \\
0 & 1 & 1 & 0 \\
0 & 0 & -1 & 0 \\
0 & 0 & 0 & -1
\end{vmatrix}=1
\]

\subsection*{3.11}
\begin{align}
  A^\tau &= A^{-1} \tag{1. Definitie orthogonale matrix}\\
  A.A^\tau &= I = A^\tau.A \tag{2. Direct gevolg van definitie}\\
  det(A^\tau) &= det(A^{-1}) \tag{3. (uit 1.)}\\
  \frac{det(A^{-1})}{det(A)} &= 1 \tag{4. (uit 3.)}\\
  det(A.A^{-1}) &= det(I) = 1 \tag{5. (uit 2.)}\\
  det(A.A^{-1}) &= det(A).det(A^{-1}) \tag{6. eigenschap determinant}\\
  \frac{det(A^{-1})}{det(A)} &= det(A).det(A^{-1}) \tag{7. (uit 4. en 5.)}\\
  \frac{1}{det(A)} &= det(A) \tag{8.}\\
  det(A) &= 1 \vee det(A) = -1 \tag{9.}
\end{align}

\section*{Hoofdstuk 4}
\subsection*{4.1}
\[
\vec{x}=\vec{p}+t(\vec{q}-\vec{p)}
\]
\subsubsection*{b)}
\[
\begin{pmatrix}
x\\y\\z
\end{pmatrix}
=
\begin{pmatrix}
1\\2\\3
\end{pmatrix}
+t
\begin{pmatrix}
1\\0\\0
\end{pmatrix}
\]
\subsubsection*{d)}
\[
\begin{pmatrix}
x\\y\\z
\end{pmatrix}
=
\begin{pmatrix}
1\\2\\3
\end{pmatrix}
+t
\begin{pmatrix}
0-1\\1-2\\2-3
\end{pmatrix}
=
\begin{pmatrix}
1\\2\\3
\end{pmatrix}
+t
\begin{pmatrix}
-1\\-1\\-1
\end{pmatrix}
\]

\subsection*{4.2}
\[
V: \vec{x} = \vec{p_0}+t_1(\vec{p_1}-\vec{p_0}) +t_2(\vec{p_2}-\vec{p_0})
\]
\subsubsection*{a)}
\[
V:\;\begin{pmatrix}
x\\y\\z
\end{pmatrix}
=
\begin{pmatrix}
1\\-1\\2
\end{pmatrix}
+t1\begin{pmatrix}
1\\2\\2
\end{pmatrix}
+t2\begin{pmatrix}
2\\1\\-1
\end{pmatrix}
\]

\subsubsection*{b)}
\[
V:\;\begin{pmatrix}
x\\y\\z
\end{pmatrix}
=
\begin{pmatrix}
1\\0\\0
\end{pmatrix}
+t1\begin{pmatrix}
-1\\1\\0
\end{pmatrix}
+t2\begin{pmatrix}
-1\\0\\1
\end{pmatrix}
\]

\subsection*{4.4}
\subsubsection*{a)}
We berekenen een normaalvector:\\
\[
\begin{pmatrix}
1\\2\\2
\end{pmatrix}
\times
\begin{pmatrix}
2\\1\\-1
\end{pmatrix}
=
\begin{pmatrix}
-4\\5\\-3
\end{pmatrix}
\]
Er bestaat een a zodat dit een eenheidsnormaal is:\\
\[
a.
\begin{pmatrix}
-4\\5\\-3
\end{pmatrix}
\]
\[
\sqrt{(-4a)^2 + (5a)^2 + (-3a)^2} = \sqrt{50a^2} = 1
\]
Antwoord:\\
\[
\begin{pmatrix}
\frac{-4}{\sqrt{50}}\\\frac{5}{\sqrt{50}}\\\frac{-3}{\sqrt{50}}
\end{pmatrix}
\]

\subsection*{4.5}
Stel:\\
\[
Normaal:\;\vec{n}=\begin{pmatrix}
a\\b\\c
\end{pmatrix}
,\;Positievector:\;\vec{p}=\begin{pmatrix}
x_0\\y_0\\z_0
\end{pmatrix},\; en \;
\vec{x}=\begin{pmatrix}
x\\y\\z
\end{pmatrix}
\]
Carthesische vergelijking van een vlak: \\
\[V:\;a(x-x_0)+b(y-y_0)+c(z-z_0)=0\]
Parametervergelijking van een vlak: \\
\[ V:\;\vec{x}=\vec{p}+t_1\vec{r_1}+t_2\vec{r_2}\]
waarbij:\\
\[\vec{r_1}=\begin{pmatrix}
-b\\a\\0
\end{pmatrix},\;en\:
\vec{r_2}=\begin{pmatrix}
-c\\0\\a
\end{pmatrix}\]

\subsubsection*{a)}
\[
V:\;2(x-1)+1(y-1)+3(z-2)=0
\]
\[
\Leftrightarrow 2x+y+3z=9
\]
\[
V:\;\begin{pmatrix}
x\\y\\z
\end{pmatrix}
=
\begin{pmatrix}
1\\1\\2
\end{pmatrix}
+t1\begin{pmatrix}
-1\\2\\0
\end{pmatrix}
+t2\begin{pmatrix}
-3\\0\\2
\end{pmatrix}
\]

\subsubsection*{b)}
\[
V:\;-1(x-2)+4(y+1)+5(z-3)=0
\]
\[
\Leftrightarrow -x+4y+5z=9
\]
\[
V:\;\begin{pmatrix}
x\\y\\z
\end{pmatrix}
=
\begin{pmatrix}
2\\-1\\3
\end{pmatrix}
+t1\begin{pmatrix}
-4\\-1\\0
\end{pmatrix}
+t2\begin{pmatrix}
-5\\0\\-1
\end{pmatrix}
\]

\subsection*{4.6}
\[
\begin{vmatrix}
x & 1 & 1 & 0\\
y & 0 & 2 & 1\\
z & 1 & 0 & 3\\
1 & 1 & 1 & 1
\end{vmatrix}
=0
\]
\[
\Leftrightarrow
x.\begin{vmatrix}
0 & 2 & 1\\
1 & 0 & 3\\
1 & 1 & 1
\end{vmatrix}
-y.\begin{vmatrix}
1 & 1 & 0\\
1 & 0 & 3\\
1 & 1 & 1
\end{vmatrix}
+z.\begin{vmatrix}
1 & 1 & 0\\
0 & 2 & 1\\
1 & 1 & 1
\end{vmatrix}
-\begin{vmatrix}
1 & 1 & 0\\
0 & 2 & 1\\
1 & 0 & 3
\end{vmatrix}
=0
\]
\[
\Leftrightarrow 5x+y+2z=7
\]

\subsection*{4.7}
\[
r_1=
\begin{pmatrix}
2\\-1\\3
\end{pmatrix}
,\;richtvector:\; r_2=
\begin{pmatrix}
2-1\\4-2\\2-3
\end{pmatrix}
=\begin{pmatrix}
1\\2\\-1
\end{pmatrix}
\]
\[
normaal:\;\begin{pmatrix}
0\\1\\0
\end{pmatrix}
\times
\begin{pmatrix}
1\\2\\-1
\end{pmatrix}
=
\begin{pmatrix}
-1\\0\\-1
\end{pmatrix}
\]
\[
V:\;-(x-1)-(z-3)=0
\]
\[
\Leftrightarrow x+z=4
\]

\subsection*{4.8}
Stel:\\
\[Positievectoren:\;\vec{p},\;en\;\vec{q}\;,\;\;Richtvector: (\vec{q}-\vec{p}),\;en\; Normalen:\; n_1=\begin{pmatrix}
a_1\\b_1\\c_1
\end{pmatrix},\;en\;n_2=\begin{pmatrix}
a_2\\b_2\\c_2
\end{pmatrix}\]
Carthesische vergelijking van een rechte:\\
\[L:\;
\left\lbrace
\begin{array}{l l}
a_1x+b_1y+c_1z=d_1\\
a_2x+b_2y+c_2z=d_2
\end{array}
\right.
\]
Parametervergelijking van een rechte:\\
\[L:\; \vec{x}=\vec{p}+t(\vec{q}-\vec{p})\]
\subsubsection*{a)}
\[L:\; \vec{x}=\begin{pmatrix}
1\\2\\8
\end{pmatrix}+t\begin{pmatrix}
3\\-1\\-4
\end{pmatrix}\]

\[\begin{pmatrix}
a\\3\\b
\end{pmatrix}=\begin{pmatrix}
1\\2\\8
\end{pmatrix}+t\begin{pmatrix}
3\\-1\\-4
\end{pmatrix}\Leftrightarrow\left\lbrace\begin{array}{l l}
a=1+3t\\
3=2+(-1)t\\
b=8-4t
\end{array}\right.\Leftrightarrow\left\lbrace\begin{array}{l l}
a=1+3(-\frac{1}{3})\\
t=-\frac{1}{3}\\
b=8-4(-\frac{1}{3})
\end{array}\right.\Leftrightarrow\left\lbrace\begin{array}{l l}
a=\frac{1}{3}\\
t=-\frac{1}{3}\\
b=12
\end{array}\right.\]
\subsubsection*{b)}
\[
V:\;3(x+4)-2(y+0)+6(z-3)=0
\]
\[
\Leftrightarrow 3x-2y+6z=6
\]
\[
\Leftrightarrow 3(1+3t)-2(2-t)+6(8-4t)=6
\]
\[
\Leftrightarrow 9t+2t-24t+3-4+48=6\;\Leftrightarrow\; 9t+2t-24t+3-4+48=6
\]
\[
\Leftrightarrow t=\frac{41}{13}
\]
Antwoord:\\
\[
\begin{pmatrix}
x\\y\\z
\end{pmatrix}=\begin{pmatrix}
1\\2\\8
\end{pmatrix}+\frac{41}{13}
\begin{pmatrix}
3\\-1\\-4
\end{pmatrix}=\begin{pmatrix}
\frac{136}{13}\\-\frac{15}{13}\\-\frac{60}{13}
\end{pmatrix}
\]

\subsection*{4.10}
\[
\begin{pmatrix}
1\\1\\1
\end{pmatrix}
\times
\begin{pmatrix}
1\\0\\1
\end{pmatrix}
=
\begin{pmatrix}
1\\0\\-1
\end{pmatrix}
\]
Is een richtvector van de rechte.
Dus:
\[
L: \begin{pmatrix}
1\\3\\2
\end{pmatrix}
+
t.\begin{pmatrix}
1\\0\\-1
\end{pmatrix}
\]
Of:\\
\[
L: \left\lbrace
\begin{array}{c c}
y &= 3\\
x+z&=3
\end{array}
\right.
\]

\subsection*{4.18}
\subsubsection*{a)}
\[
\begin{pmatrix}
1\\3\\-2
\end{pmatrix}
\times
\begin{pmatrix}
0\\3\\1
\end{pmatrix}
=
\begin{pmatrix}
9\\-1\\3
\end{pmatrix}
\]
\[
\begin{pmatrix}
0\\3\\1
\end{pmatrix}
\times
\begin{pmatrix}
1\\3\\-2
\end{pmatrix}
=
\begin{pmatrix}
-9\\1\\-3
\end{pmatrix}
\]

\subsubsection*{d)}
\[
\left(
\begin{pmatrix}
1\\3\\-2
\end{pmatrix}
\times
\begin{pmatrix}
0\\-1\\2
\end{pmatrix}
\right)
.
\begin{pmatrix}
0\\3\\1
\end{pmatrix}
=
\begin{pmatrix}
8\\-2\\1
\end{pmatrix}
.
\begin{pmatrix}
0\\3\\1
\end{pmatrix}
=-7
\]

\subsubsection*{e)}
\[
\begin{pmatrix}
1\\3\\-2
\end{pmatrix}
\times
\left(
\begin{pmatrix}
0\\3\\1
\end{pmatrix}
\times
\begin{pmatrix}
0\\-1\\2
\end{pmatrix}
\right)
=
\begin{pmatrix}
1\\3\\-2
\end{pmatrix}
\times
\begin{pmatrix}
7\\0\\0
\end{pmatrix}
=
\begin{pmatrix}
0\\14\\-21
\end{pmatrix}
\]

\subsection*{4.22}
\subsubsection*{a)}
\[
d(\vec{p},\vec{v})) = \frac{|10-2.1-3.1-(-1).2|}{\sqrt{2^2 + 3^3 + 1^2}} = \frac{7}{\sqrt{14}}
\]

\subsubsection*{b)}
\[
\vec{r_1} = 
\begin{pmatrix}
-1\\0\\1
\end{pmatrix}
,\;\;
\vec{r_2} = 
\begin{pmatrix}
-1\\1\\0
\end{pmatrix}
,\;\;
\vec{p} = 
\begin{pmatrix}
1\\0\\0
\end{pmatrix}
\]
\[
\vec{r_1}\times\vec{r_2}=
\begin{pmatrix}
-1\\0\\1
\end{pmatrix}
\times
\begin{pmatrix}
-1\\1\\0
\end{pmatrix}
=
\begin{pmatrix}
-1\\-1\\-1
\end{pmatrix}
\]
\[-(x-1)-y-z=0\]
\[x+y+z=1\]
afstand tot $(2,1,1)^\tau$:\\
\[
d(\vec{p},v) = \frac{|1-2-1-1|}{\sqrt{3}} = \sqrt{3}
\]

\subsection*{4.30}
\subsubsection*{a)}
$\left\vert det\begin{pmatrix}
1 & -1 & 2 \\
1 & 4 & 2 \\
0 & 0 & 2
\end{pmatrix} \right\vert$

\section*{Hoofdstuk 5}
\subsection*{5.1}
\subsubsection*{a)}
\[
\begin{vmatrix}
2-\lambda & 2 \\
1 & 3-\lambda
\end{vmatrix}
=0
\;\;\;\longrightarrow\;\;\;
\lambda^2 - (2+3)\lambda +4 = 0
\;\;\;\longrightarrow\;\;\;
\lambda = 4 \;\vee\; \lambda = 1
\]
Voor $\lambda = 4$: 
\[
\left(
\begin{array}{cc|c}
-2 & 2 & 0 \\
1 & -1 & 0
\end{array}
\right)
\;\;\;\longrightarrow\;\;\;
\left(
\begin{array}{cc|c}
0 & 0 & 0 \\
1 & -1 & 0
\end{array}
\right)
\;\;\;\longrightarrow\;\;\;
eigenvector:\;
c
\begin{pmatrix}
1\\1
\end{pmatrix}
\]
Voor $\lambda = 1$: 
\[
\left(
\begin{array}{cc|c}
1 & 2 & 0 \\
1 & 2 & 0
\end{array}
\right)
\;\;\;\longrightarrow\;\;\;
\left(
\begin{array}{cc|c}
1 & 2 & 0 \\
0 & 0 & 0
\end{array}
\right)
\;\;\;\longrightarrow\;\;\;
eigenvector:\;
c
\begin{pmatrix}
-2\\1
\end{pmatrix}
\]

\subsubsection*{b)}
\[
\begin{vmatrix}
3-\lambda & 1 \\
1 & 3-\lambda
\end{vmatrix}
=0
\;\;\;\longrightarrow\;\;\;
\lambda^2 - (3+3)\lambda + 8 = 0
\;\;\;\longrightarrow\;\;\;
\lambda = 4 \;\vee\; \lambda = 2
\]
Voor $\lambda = 4$: 
\[
\left(
\begin{array}{cc|c}
-1 & 1 & 0 \\
1 & -1 & 0
\end{array}
\right)
\;\;\;\longrightarrow\;\;\;
\left(
\begin{array}{cc|c}
-1 & 1 & 0 \\
0 & 0 & 0
\end{array}
\right)
\;\;\;\longrightarrow\;\;\;
eigenvector:\;
c
\begin{pmatrix}
1\\1
\end{pmatrix}
\]
Voor $\lambda = 2$: 
\[
\left(
\begin{array}{cc|c}
1 & 1 & 0 \\
1 & 1 & 0
\end{array}
\right)
\;\;\;\longrightarrow\;\;\;
\left(
\begin{array}{cc|c}
1 & 1 & 0 \\
0 & 0 & 0
\end{array}
\right)
\;\;\;\longrightarrow\;\;\;
eigenvector:\;
c
\begin{pmatrix}
-1\\1
\end{pmatrix}
\]
\subsubsection*{e)}
\[
\begin{vmatrix}
1-\lambda & 2 & 0 \\
2 & 1-\lambda & 0 \\
0 & 2 & -1-\lambda
\end{vmatrix}
=0
\;\;\;\longrightarrow\;\;\;
(1-\lambda)(\lambda^2-1^2) - 2(2(-1-\lambda)) = 0
\;\;\;\longrightarrow\;\;\;
\lambda = -1 \;\vee\; \lambda = 3
\]
Voor $\lambda = -1$: 
\[
\left(
\begin{array}{ccc|c}
2 & 2 & 0 & 0\\
2 & 2 & 0 & 0\\
0 & 2 & 0 & 0
\end{array}
\right)
\;\;\;\longrightarrow\;\;\;
\left(
\begin{array}{ccc|c}
1 & 1 & 0 & 0 \\
0 & 0 & 0 & 0 \\
0 & 1 & 0 & 0
\end{array}
\right)
\;\;\;\longrightarrow\;\;\;
eigenvector:\;
c
\begin{pmatrix}
0\\0\\1
\end{pmatrix}
\]
Voor $\lambda = 3$: 
\[
\left(
\begin{array}{ccc|c}
-2 & 2 & 0 & 0\\
2 & -2 & 0 & 0\\
0 & 2 & -4 & 0
\end{array}
\right)
\;\;\;\longrightarrow\;\;\;
\left(
\begin{array}{ccc|c}
-1 & 1 & 0 & 0 \\
0 & 0 & 0 & 0 \\
0 & 1 & -2 & 0
\end{array}
\right)
\;\;\;\longrightarrow\;\;\;
eigenvector:\;
c
\begin{pmatrix}
2\\2\\1
\end{pmatrix}
\]

\subsubsection*{f)}
\[
\begin{vmatrix}
-\lambda & 2 & 2 \\
2 & -\lambda & 2 \\
2 & 2 & -\lambda
\end{vmatrix}
=0
\;\;\;\longrightarrow\;\;\;
-\lambda(\lambda^2-4)-2(-2\lambda-4)+2(4+2\lambda)=0
\;\;\;\longrightarrow\;\;\;
\lambda = -2 \;\vee\; \lambda = 4
\]
Voor $\lambda = -2$
\[
\left(
\begin{array}{ccc|c}
2 & 2 & 2 & 0\\
2 & 2 & 2 & 0\\
2 & 2 & 2 & 0
\end{array}
\right)
\;\;\;\longrightarrow\;\;\;
\left(
\begin{array}{ccc|c}
1 & 1 & 1 & 0 \\
0 & 0 & 0 & 0 \\
0 & 0 & 0 & 0
\end{array}
\right)
\;\;\;\longrightarrow\;\;\;
eigenvector:\;
c_1
\begin{pmatrix}
-1\\1\\0
\end{pmatrix}
+
c_2
\begin{pmatrix}
-1\\0\\1
\end{pmatrix}
\]
Voor $\lambda = 4$
\[
\left(
\begin{array}{ccc|c}
-4 & 2 & 2 & 0\\
2 & -4 & 2 & 0\\
0 & 2 & -4 & 0
\end{array}
\right)
\;\;\;\longrightarrow\;\;\;
\left(
\begin{array}{ccc|c}
1 & 0 & -1 & 0 \\
0 & 1 & -1 & 0 \\
0 & 0 & 0 & 0
\end{array}
\right)
\;\;\;\longrightarrow\;\;\;
eigenvector:\;
c
\begin{pmatrix}
1\\1\\1
\end{pmatrix}
\]

\subsubsection*{g)}
\[
\begin{vmatrix}
-\lambda & 3 & 0 \\
3 & -\lambda & 3 \\
0 & 3 & -\lambda
\end{vmatrix}
=0
\;\;\;\longrightarrow\;\;\;
-\lambda(\lambda^2-9)-3(-3\lambda)=0
\;\;\;\longrightarrow\;\;\;
\lambda = 0 \;\vee\; \lambda = 3\sqrt{2} \;\vee\; \lambda = -3\sqrt{2}
\]
Voor $\lambda = 0$
\[
\left(
\begin{array}{ccc|c}
0 & 3 & 0 & 0\\
3 & 0 & 3 & 0\\
0 & 3 & 0 & 0
\end{array}
\right)
\;\;\;\longrightarrow\;\;\;
\left(
\begin{array}{ccc|c}
1 & 0 & 1 & 0 \\
0 & 1 & 0 & 0 \\
0 & 0 & 0 & 0
\end{array}
\right)
\;\;\;\longrightarrow\;\;\;
eigenvector:\;
c
\begin{pmatrix}
-1\\0\\1
\end{pmatrix}
\]
Voor $\lambda = 3\sqrt{2}$
\[
\left(
\begin{array}{ccc|c}
-3\sqrt{2} & 3 & 0 & 0\\
3 & -3\sqrt{2} & 3 & 0\\
0 & 3 & -3\sqrt{2} & 0
\end{array}
\right)
\;\;\;\longrightarrow\;\;\;
\left(
\begin{array}{ccc|c}
1 & 0 & -1 & 0 \\
0 & 1 & -\sqrt{2} & 0 \\
0 & 0 & 0 & 0
\end{array}
\right)
\;\;\;\longrightarrow\;\;\;
eigenvector:\;
c
\begin{pmatrix}
1\\\sqrt{2}\\1
\end{pmatrix}
\]
Voor $\lambda = -3\sqrt{2}$
\[
\left(
\begin{array}{ccc|c}
3\sqrt{2} & 3 & 0 & 0\\
3 & 3\sqrt{2} & 3 & 0\\
0 & 3 & 3\sqrt{2} & 0
\end{array}
\right)
\;\;\;\longrightarrow\;\;\;
\left(
\begin{array}{ccc|c}
1 & 0 & -1 & 0 \\
0 & 1 & \sqrt{2} & 0 \\
0 & 0 & 0 & 0
\end{array}
\right)
\;\;\;\longrightarrow\;\;\;
eigenvector:\;
c
\begin{pmatrix}
1\\-\sqrt{2}\\1
\end{pmatrix}
\]

\subsection*{5.2}
\subsubsection*{a)}
TODO
\subsubsection*{a) voor 5.1g}
\[
c
\begin{pmatrix}
-1\\0\\1
\end{pmatrix}
 \overset{
 \sqrt{(-1a)^2 + (0a)^2 + a^2} = 1}{\rightarrow} 
 c
\begin{pmatrix}
-\frac{\sqrt{2}}{2}\\0\\\frac{\sqrt{2}}{2}
\end{pmatrix}
\]
\[
c
\begin{pmatrix}
1\\\sqrt{2}\\1
\end{pmatrix}
 \overset{
 \sqrt{a^2 + (\sqrt{2}a)^2 + a^2} = 1}{\rightarrow} 
 c
\begin{pmatrix}
\frac{1}{2}\\\frac{\sqrt{2}}{2}\\\frac{1}{2}
\end{pmatrix}
\]
\[
c
\begin{pmatrix}
1\\-\sqrt{2}\\1
\end{pmatrix}
 \overset{
 \sqrt{a^2 + (-\sqrt{2}a)^2 + a^2} = 1}{\rightarrow} 
 c
\begin{pmatrix}
\frac{1}{2}\\-\frac{\sqrt{2}}{2}\\\frac{1}{2}
\end{pmatrix}
\]
\subsubsection*{b) voor 5.1g}
\[
\begin{pmatrix}
0 & 3 & 0 \\
3 & 0 & 3 \\
0 & 3 & 0
\end{pmatrix}^2
 \overset{!}{\neq} 
 \begin{pmatrix}
1 & 0 & 0 \\
0 & 1 & 0 \\
0 & 0 & 1
\end{pmatrix}
\]
TODO

\subsection*{5.4}
\[
det(A-\lambda I)=det(A) = 0
\]

\subsection*{5.5}
\[\left(
\begin{pmatrix}
a & b \\
0 & d
\end{pmatrix}
-\lambda I \right)\vec{x} = \vec{0}
\]
\[
\Leftrightarrow
\begin{vmatrix}
a-\lambda & b \\
0 & d-\lambda
\end{vmatrix}
=0
\]
\[
\Leftrightarrow
(a-\lambda)(d-\lambda) = 0
\]
\[
\Leftrightarrow
\lambda=a \vee \lambda=d
\]
TODO BEWIJS DOOR INDUCTIE

\subsection*{5.6}
\subsubsection*{a)}
\[
D=
\begin{pmatrix}
4 & 0\\
0 & 1\\
\end{pmatrix}
\;\;\;
P=
\begin{pmatrix}
1 & -2\\
1 & 1\\
\end{pmatrix}
\;\;\;
P^{-1}=
\begin{pmatrix}
\frac{1}{3} & \frac{2}{3}\\
-\frac{1}{3} & \frac{1}{3}\\
\end{pmatrix}
\]
\[
\begin{pmatrix}
\frac{1}{3} & \frac{2}{3}\\
-\frac{1}{3} & \frac{1}{3}\\
\end{pmatrix}
.
\begin{pmatrix}
2 & 2\\
1 & 3\\
\end{pmatrix}
.
\begin{pmatrix}
1 & -2\\
1 & 1\\
\end{pmatrix}
 \overset{!}{=} 
\begin{pmatrix}
4 & 0\\
0 & 1\\
\end{pmatrix}
\]
Diagonaliseerbaar.

\subsection*{5.7}
\subsubsection*{a)}
TODO

\subsubsection*{b)}
\[
\begin{vmatrix}
1-\lambda & 0 & 0\\
0 & 1- \lambda & 1 \\
0 & 1 & 1-\lambda
\end{vmatrix}
=0
\;\;\;\longrightarrow\;\;\;
(1-\lambda)((1-\lambda)^2-1)=0
\;\;\;\longrightarrow\;\;\;
\lambda = 0 \;\vee\; \lambda = 1 \;\vee\; \lambda = 2
\]
\[
D = 
\begin{pmatrix}
0 & 0 & 0\\
0 & 1 & 0\\
0 & 0 & 2
\end{pmatrix}
\]
Voor $\lambda = 0$: 
\[
\left(
\begin{array}{ccc|c}
1 & 0 & 0 & 0\\
0 & 1 & 1 & 0\\
0 & 0 & 0 & 0
\end{array}
\right)
\;\;\;\longrightarrow\;\;\;
eigenvector:\;
c
\begin{pmatrix}
0\\-1\\1
\end{pmatrix}
\]
Voor $\lambda = 1$: 
\[
\left(
\begin{array}{ccc|c}
0 & 0 & 0 & 0\\
0 & 0 & 1 & 0\\
0 & 1 & 0 & 0
\end{array}
\right)
\;\;\;\longrightarrow\;\;\;
eigenvector:\;
c
\begin{pmatrix}
1\\0\\0
\end{pmatrix}
\]
Voor $\lambda = 2$: 
\[
\left(
\begin{array}{ccc|c}
1 & 0 & 0 & 0\\
0 & 1 & -1 & 0\\
0 & 0 & 0 & 0
\end{array}
\right)
\;\;\;\longrightarrow\;\;\;
eigenvector:\;
c
\begin{pmatrix}
0\\1\\1
\end{pmatrix}
\]
\[
P = 
\begin{pmatrix}
0 & 1 & 0\\
-1 & 0 & 1\\
1 & 0 & 1
\end{pmatrix}
\;\;\;
P^{-1} = 
\begin{pmatrix}
0 & -\frac{1}{2} & \frac{1}{2}\\
1 & 0 & 0\\
0 & \frac{1}{2} & \frac{1}{2}
\end{pmatrix}
\]
\[
\begin{pmatrix}
0 & -\frac{1}{2} & \frac{1}{2}\\
1 & 0 & 0\\
0 & \frac{1}{2} & \frac{1}{2}
\end{pmatrix}
.
\begin{pmatrix}
1 & 0 & 0\\
0 & 1 & 1\\
0 & 1 & 1
\end{pmatrix}
.
\begin{pmatrix}
0 & 1 & 0\\
-1 & 0 & 1\\
1 & 0 & 1
\end{pmatrix}
 \overset{!}{=} 
\begin{pmatrix}
0 & 0 & 0\\
0 & 1 & 0\\
0 & 0 & 2
\end{pmatrix}
\]
Diagonaliseerbaar.

\subsubsection*{c)}
Niet diagonaliseerbaar.

\subsubsection*{e)}
\[
\begin{vmatrix}
-\lambda & 1 & 0\\
0 & - \lambda & 1 \\
1 & 0 & -\lambda
\end{vmatrix}
=0
\;\;\;\longrightarrow\;\;\;
(-\lambda)^3 + 1 = 0
\;\;\;\longrightarrow\;\;\;
\lambda = 1 \;\vee\; \lambda = -\frac{1}{2}-\frac{\sqrt{3}}{2}i \;\vee\; \lambda = -\frac{1}{2}+\frac{\sqrt{3}}{2}i
\]
\[
D = 
\begin{pmatrix}
1 & 0 & 0\\
0 & -\frac{1}{2}-\frac{\sqrt{3}}{2}i & 0\\
0 & 0 & -\frac{1}{2}+\frac{\sqrt{3}}{2}i
\end{pmatrix}
\]
TODO

\subsubsection*{e)}
Zie toledo\\
\url{https://cygnus.cc.kuleuven.be/bbcswebdav/pid-11087333-dt-content-rid-10466789_2/courses/a-G0O17a-1213/Oefening%205_7_e.pdf}

\subsection*{5.8}
\subsubsection*{a)}
\[
\begin{vmatrix}
1+i-\lambda & 2-i \\
3+i & -i-\lambda
\end{vmatrix}
=0
\;\;\;\longrightarrow\;\;\;
\]
\[
(1+i-\lambda)(-i-\lambda)-(2-i)(3+i)=0
\;\;\;\longrightarrow\;\;\;
\lambda = 3 \;\vee\; \lambda = -2
\]
Voor $\lambda = 3$: 
\[
\left(
\begin{array}{cc|c}
4+i & 2-i & 0 \\
3+i & -3-i & 0
\end{array}
\right)
\;\;\;\longrightarrow\;\;\;
\left(
\begin{array}{cc|c}
1 & -1 & 0 \\
0 & 0 & 0
\end{array}
\right)
\;\;\;\longrightarrow\;\;\;
eigenvector:\;
c
\begin{pmatrix}
1\\1
\end{pmatrix}
\]
Voor $\lambda = -2$: 
\[
\left(
\begin{array}{cc|c}
3+i & 2-i & 0 \\
3+i & 2-i & 0
\end{array}
\right)
\;\;\;\longrightarrow\;\;\;
\left(
\begin{array}{cc|c}
3+i & 2+i & 0 \\
0 & 0 & 0
\end{array}
\right)
\;\;\;\longrightarrow\;\;\;
eigenvector:\;
c
\begin{pmatrix}
\frac{-i-7}{10}\\1
\end{pmatrix}
\]

\subsubsection*{b)}
\[
\begin{vmatrix}
-\lambda & -i & 0\\
i & - \lambda & -i \\
0 & i & -\lambda
\end{vmatrix}
=0
\;\;\;\longrightarrow\;\;\;
-\lambda((-\lambda)^2 + i^2)-i(i\lambda)
\;\;\;\longrightarrow\;\;\;
\lambda = 0 \;\vee\; \lambda = \sqrt{2} \;\vee\; \lambda = -\sqrt{2}
\]
\[
D = 
\begin{pmatrix}
0 & 0 & 0\\
0 & \sqrt{2} & 0\\
0 & 0 & -\sqrt{2}
\end{pmatrix}
\]
Voor $\lambda = 0$: 
\[
\left(
\begin{array}{ccc|c}
0 & -i & 0 & 0\\
i & 0 & -i & 0\\
0 & i & 0 & 0\\
\end{array}
\right)
\;\;\;\longrightarrow\;\;\;
\left(
\begin{array}{ccc|c}
0 & 0 & 0 & 0\\
1 & 0 & -1 & 0\\
0 & 1 & 0 & 0
\end{array}
\right)
\;\;\;\longrightarrow\;\;\;
eigenvector:\;
c
\begin{pmatrix}
1\\0\\1
\end{pmatrix}
\]
Voor $\lambda = \sqrt{2}$: 
\[
\left(
\begin{array}{ccc|c}
-\sqrt{2} & -i & 0 & 0\\
i & -\sqrt{2} & -i & 0\\
0 & i & -\sqrt{2} & 0\\
\end{array}
\right)
\;\;\;\longrightarrow\;\;\;
TODO
\]
Voor $\lambda = -\sqrt{2}$: 
\[
\left(
\begin{array}{ccc|c}
\sqrt{2} & -i & 0 & 0\\
i & \sqrt{2} & -i & 0\\
0 & i & \sqrt{2} & 0\\
\end{array}
\right)
\;\;\;\longrightarrow\;\;\;
TODO
\]

\subsubsection*{c)}
\[
\begin{vmatrix}
2-\lambda & i\\
i & 2- \lambda \\
\end{vmatrix}
=0
\;\;\;\longrightarrow\;\;\;
(2-\lambda)(1-\lambda)+1=0
\;\;\;\longrightarrow\;\;\;
\lambda = \frac{3+i\sqrt{3}}{2} \;\vee\; \lambda = \frac{3-i\sqrt{3}}{2}
\]
\[
D = 
\begin{pmatrix}
\frac{3+i\sqrt{3}}{2} & 0\\
0 & \frac{3-i\sqrt{3}}{2}\\
\end{pmatrix}
\]
Voor $\lambda = \frac{3+i\sqrt{3}}{2}$: 
\[
\left(
\begin{array}{cc|c}
2-\frac{3+i\sqrt{3}}{2} & i & 0 \\
i & 1-\frac{3+i\sqrt{3}}{2} & 0
\end{array}
\right)
\;\;\;\longrightarrow\;\;\;
TODO
\]
Voor $\lambda = \frac{3-i\sqrt{3}}{2}$: 
\[
\left(
\begin{array}{cc|c}
2-\frac{3-i\sqrt{3}}{2} & i & 0 \\
i & 1-\frac{3-i\sqrt{3}}{2} & 0
\end{array}
\right)
\;\;\;\longrightarrow\;\;\;
TODO
\]

\subsection*{5.11}
\subsubsection*{a)}
\[
\begin{vmatrix}
\alpha-\beta & \beta & \beta \\
\beta & \alpha-\beta & \beta \\
\beta & \beta & \alpha-\beta
\end{vmatrix}
=0
\longrightarrow
(-\lambda-\alpha-2\beta)(\lambda-\alpha+\beta)^2 = 0
\]

\subsection*{5.13}
1. Basisstap
Het geldt voor $k=1$ want (gegeven)
\[
A^k = XD^kX^{-1}
\]
2. Inductiestap: tel dat het klopt voor $k=n$ met $n>1$ en $n \in \mathrm{N}$
3. Te bewijzen: 
\[
A^{k+1} = X.D^{k+1}.X^{-1}
\]
\[
A.A^k = X.D.D^k.X^{-1}
\]
\[
A.A^k = X.D.X^{-1}\;.\;XD^kX^{-1} = X.D.D^k.X^{-1}
\]
QED

\subsection*{5.15}
TODO

\section*{Hoofdstuk 6}
\subsection*{6.1}
\[
\frac{d\begin{pmatrix}
e^{2t}\\2e^{2t}
\end{pmatrix}}{dt}
=
\begin{pmatrix}
2e^{2t}\\4e^{2t}
\end{pmatrix}
\;\;\;
\frac{d\begin{pmatrix}
e^{3t}\\e^{3t}
\end{pmatrix}}{dt}
=
\begin{pmatrix}
3e^{3t}\\3e^{3t}
\end{pmatrix}
\]
\[
\begin{pmatrix}
4 & -1 \\
2 & 1
\end{pmatrix}
.
\begin{pmatrix}
e^{2t}\\2e^{2t}
\end{pmatrix}
=
\begin{pmatrix}
2e^{2t}\\4e^{2t}
\end{pmatrix}
\]
OK
\[
\begin{pmatrix}
4 & -1 \\
2 & 1
\end{pmatrix}
.
\begin{pmatrix}
e^{3t}\\e^{3t}
\end{pmatrix}
=
\begin{pmatrix}
3e^{3t}\\3e^{3t}
\end{pmatrix}
\]
OK

\[
\begin{vmatrix}
4-\lambda & -1 \\
2 & 1-\lambda
\end{vmatrix}
=0
\;\;\;\longrightarrow\;\;\;
(4-\lambda)(1-\lambda)+2=0
\;\;\;\longrightarrow\;\;\;
\lambda = 2 \;\;\;\vee\;\;\;\lambda = 3
\]
Voor $\lambda = 2$: 
\[
\left(
\begin{array}{cc|c}
2 & -1 & 0 \\
2 & -1 & 0
\end{array}
\right)
\;\;\;\longrightarrow\;\;\;
\left(
\begin{array}{cc|c}
2 & -1 & 0 \\
0 & 0 & 0
\end{array}
\right)
\;\;\;\longrightarrow\;\;\;
eigenvector:\;
c
\begin{pmatrix}
1\\2
\end{pmatrix}
\]
Voor $\lambda = 3$: 
\[
\left(
\begin{array}{cc|c}
1 & -1 & 0 \\
2 & -2 & 0
\end{array}
\right)
\;\;\;\longrightarrow\;\;\;
\left(
\begin{array}{cc|c}
1 & -1 & 0 \\
0 & 0 & 0
\end{array}
\right)
\;\;\;\longrightarrow\;\;\;
eigenvector:\;
c
\begin{pmatrix}
1\\1
\end{pmatrix}
\]
\[
\vec{x} = c_1e^{2t}\begin{pmatrix}
1\\2
\end{pmatrix}+c_2e^{3t}\begin{pmatrix}
1\\1
\end{pmatrix}
\]

\subsection*{6.2}
Stabiliteit:
\[
\lambda_1 \ge 0 \;\;\;of\;\;\; \lambda_2 \ge 0
\]
\subsubsection*{a)}
\[
\vec{x'} = 
\begin{pmatrix}
5 & 4\\
-1 & 0
\end{pmatrix}
\vec{x}
\]
\[
\begin{vmatrix}
5-\lambda & 4 \\
-1 & -\lambda
\end{vmatrix}
=0
\;\;\;\longrightarrow\;\;\;
(5-\lambda)(-\lambda)+4=0
\;\;\;\longrightarrow\;\;\;
\lambda = 4 \;\;\;\vee\;\;\;\lambda = 1
\]
Voor $\lambda = 4$: 
\[
\left(
\begin{array}{cc|c}
1 & 4 & 0 \\
-1 & -4 & 0
\end{array}
\right)
\;\;\;\longrightarrow\;\;\;
\left(
\begin{array}{cc|c}
1 & 4 & 0 \\
0 & 0 & 0
\end{array}
\right)
\;\;\;\longrightarrow\;\;\;
eigenvector:\;
c
\begin{pmatrix}
-4\\1
\end{pmatrix}
\]
Voor $\lambda = 1$: 
\[
\left(
\begin{array}{cc|c}
4 & 4 & 0 \\
-1 & -1 & 0
\end{array}
\right)
\;\;\;\longrightarrow\;\;\;
\left(
\begin{array}{cc|c}
1 & 1 & 0 \\
0 & 0 & 0
\end{array}
\right)
\;\;\;\longrightarrow\;\;\;
eigenvector:\;
c
\begin{pmatrix}
-1\\1
\end{pmatrix}
\]
\[
\vec{x} = c_1e^{4t}\begin{pmatrix}
-4\\1
\end{pmatrix}+c_2e^{t}\begin{pmatrix}
-1\\1
\end{pmatrix}
\]
Evenwichten:
\[
5x+4y=0 \;\;\;en\;\;\; -x=0
\]
\[
x=0 \;\;\;en\;\;\; y=0
\]
De oorsprong is een onstabiel evenwicht.

\subsubsection*{b)}
\[
\vec{x}' =
\begin{pmatrix}
1 & 1\\
4 & 1\\
\end{pmatrix} 
\vec{x}
\]
\[
\begin{vmatrix}
1-\lambda & 1\\
4 & 1-\lambda
\end{vmatrix}
= 0
\longrightarrow
\lambda = -1 \;\;\;\vee\;\;\; \lambda =3
\]
Voor $\lambda = -1$: 
\[
\left(
\begin{array}{cc|c}
2 & 1 & 0 \\
4 & 2 & 0
\end{array}
\right)
\;\;\;\longrightarrow\;\;\;
\left(
\begin{array}{cc|c}
2 & 1 & 0 \\
0 & 0 & 0
\end{array}
\right)
\;\;\;\longrightarrow\;\;\;
eigenvector:\;
c
\begin{pmatrix}
1\\-2
\end{pmatrix}
\]
Voor $\lambda = 3$: 
\[
\left(
\begin{array}{cc|c}
-2 & 1 & 0 \\
4 & -2 & 0
\end{array}
\right)
\;\;\;\longrightarrow\;\;\;
\left(
\begin{array}{cc|c}
-2 & 1 & 0 \\
0 & 0 & 0
\end{array}
\right)
\;\;\;\longrightarrow\;\;\;
eigenvector:\;
c
\begin{pmatrix}
1\\2
\end{pmatrix}
\]
\[
\vec{x} = c_1e^{-t}\begin{pmatrix}
1\\-2
\end{pmatrix}+c_2e^{3t}\begin{pmatrix}
1\\2
\end{pmatrix}
\]
Evenwichten:\\
\[
x+y=0 \;\;\;en\;\;\; x+y=0
\]
\[
x=0 \;\;\;en\;\;\; y=0
\]
De oorsprong is een onstabie evenwicht.

\subsubsection*{c)}
\[
\vec{x'} = \begin{pmatrix}
4 & -2\\
5 & -2
\end{pmatrix}\vec{x}
\]
\[
\begin{vmatrix}
4-\lambda & -2\\
5 & -2-\lambda
\end{vmatrix}
=0
\longrightarrow
\lambda = 1 \pm i
\]
voor $\lambda=1+i$:
\[
\left(
\begin{array}{cc|c}
3-i & -2 & 0 \\
5 & -3-i & 0
\end{array}
\right)
\longrightarrow
c
\begin{pmatrix}
2 \\ 3+i
\end{pmatrix}
\]
voor $\lambda=1ii$:
\[
\left(
\begin{array}{cc|c}
3+i & -2 & 0 \\
5 & -3+i & 0
\end{array}
\right)
\longrightarrow
c
\begin{pmatrix}
2 \\ 3-i
\end{pmatrix}
\]
\[
x(t)=
c_1e^t(cos(t)+isin(t))
\begin{pmatrix}
2 \\ 3+i
\end{pmatrix}
+
c_2e^t(cos(-t)+isin(-t))
\begin{pmatrix}
2 \\ 3-i
\end{pmatrix}
\]
\[
\begin{pmatrix}
1\\2
\end{pmatrix}
=
c_1
e^{(1-i)t}
\begin{pmatrix}
2\\3+i
\end{pmatrix}
+
c_2
e^{(1+i)t}
\begin{pmatrix}
2\\3-i
\end{pmatrix}
\longrightarrow
c_1 = \frac{1}{2}
\;\;\;en\;\;\;
c_2 = \frac{1}{2}
\]
\[
\vec{x'} = e^t\begin{pmatrix}
cos(t)-sin(t)\\2cos(t)-sin(t)
\end{pmatrix}
\]


\subsection*{6.3}
\subsubsection*{a)}
\[
\vec{x'} = 
\begin{pmatrix}
4 & 2\\
3 & 3
\end{pmatrix}
\vec{x}
\]
\[
\begin{vmatrix}
4-\lambda & 2 \\
3 & 3-\lambda
\end{vmatrix}
=0
\;\;\;\longrightarrow\;\;\;
(4-\lambda)(3-\lambda)-6=0
\;\;\;\longrightarrow\;\;\;
\lambda = 6 \;\;\;\vee\;\;\;\lambda = 1
\]
Voor $\lambda = 6$: 
\[
\left(
\begin{array}{cc|c}
-2 & 2 & 0 \\
3 & -3 & 0
\end{array}
\right)
\;\;\;\longrightarrow\;\;\;
\left(
\begin{array}{cc|c}
-1 & 1 & 0 \\
0 & 0 & 0
\end{array}
\right)
\;\;\;\longrightarrow\;\;\;
eigenvector:\;
c
\begin{pmatrix}
1\\1
\end{pmatrix}
\]
Voor $\lambda = 1$: 
\[
\left(
\begin{array}{cc|c}
3 & 2 & 0 \\
3 & 2 & 0
\end{array}
\right)
\;\;\;\longrightarrow\;\;\;
\left(
\begin{array}{cc|c}
3 & 2 & 0 \\
0 & 0 & 0
\end{array}
\right)
\;\;\;\longrightarrow\;\;\;
eigenvector:\;
c
\begin{pmatrix}
2\\-3
\end{pmatrix}
\]
\[
\vec{x} = c_1e^{6t}\begin{pmatrix}
1\\1
\end{pmatrix}+c_2e^{t}\begin{pmatrix}
2-3
\end{pmatrix}
\]

\subsubsection*{c)}
\[
\vec{x'}
= 
\begin{pmatrix}
6 & -8\\
4 é -6
\end{pmatrix}
\vec{x}
\]
\[
\begin{vmatrix}
6-\lambda & -8\\
4 & -6-\lambda
\end{vmatrix}
=0
\longrightarrow
\lambda= \pm 2
\]
voor $\lambda=2$:
\[
\left(
\begin{array}{cc|c}
4 & -8 & 0 \\
4 & -8 & 0
\end{array}
\right)
\;\;\;\longrightarrow\;\;\;
\left(
\begin{array}{cc|c}
1 & -2 & 0 \\
0 & 0 & 0
\end{array}
\right)
\;\;\;\longrightarrow\;\;\;
c
\begin{pmatrix}
2\\1
\end{pmatrix}
\]
voor $\lambda=-2$:
\[
\left(
\begin{array}{cc|c}
8 & -8 & 0 \\
4 & -4 & 0
\end{array}
\right)
\;\;\;\longrightarrow\;\;\;
\left(
\begin{array}{cc|c}
1 & -1 & 0 \\
0 & 0 & 0
\end{array}
\right)
\;\;\;\longrightarrow\;\;\;
c
\begin{pmatrix}
1\\1
\end{pmatrix}
\]
\[
\begin{pmatrix}
1\\2
\end{pmatrix}
=
c_1
\begin{pmatrix}
2\\1
\end{pmatrix}
+
c_2
\begin{pmatrix}
1\\1
\end{pmatrix}
\;\;\;\longrightarrow\;\;\;
c_1 = 1
\;\;\;en\;\;\;
c_2 = -2
\]
\[
\vec{x}=
e^{2t}
\begin{pmatrix}
2\\1
\end{pmatrix}
+
2e^{-2t}
\begin{pmatrix}
1\\1
\end{pmatrix}
\]

\subsection*{6.5}
\subsubsection*{a)}
\[
-k_1a+k_2b+k_1a-(k_2+k_3)b+k_3b=0
\]
OK
\subsubsection*{b)}
\[
\vec{x'}=
\begin{pmatrix}
-2 & 1 & 0\\
2 & -3 & 0\\
0 & 2 & 0
\end{pmatrix}
\vec{x}
\]
\[
\begin{vmatrix}
-2-\lambda & 1 & 0\\
2 & -3-\lambda & 0\\
0 & 2 & -\lambda
\end{vmatrix}
\longrightarrow
\lambda=0 \vee \lambda=-1 \vee \lambda=-4
\]
voor $\lambda=0$:
\[
c
\begin{pmatrix}
1\\1\\-2
\end{pmatrix}
\]
voor $\lambda=-1$:
\[
c
\begin{pmatrix}
2\\-1\\-1
\end{pmatrix}
\]
voor $\lambda=-4$:
\[
c
\begin{pmatrix}
-1\\2\\-1
\end{pmatrix}
\]

\subsection*{6.6}
\[
\left\lbrace
\begin{array}{c c}
x' = 2x+y\\
y' = 3x+4y\\
z' = 5x-6y+32
\end{array}
\right.
\]
\[
p(\lambda) =
\begin{vmatrix}
2-\lambda & 1 & 0\\
3 & 4-\lambda & 0\\
5 & -6 & 3-\lambda
\end{vmatrix}=0
\longrightarrow
\lambda=1 \vee \lambda=3 \vee \lambda=5
\]
voor $\lambda=1$:
\[
c
\begin{pmatrix}
-2 \\ 2 \\ 11
\end{pmatrix}
\]
voor $\lambda=3$:
\[
c
\begin{pmatrix}
0 \\ 0 \\ 1
\end{pmatrix}
\]
voor $\lambda=5$:
\[
c
\begin{pmatrix}
2 \\ 6 \\ -13
\end{pmatrix}
\]
\[
\begin{pmatrix}
5 \\ 4 \\ 3
\end{pmatrix}
=
c_1
\begin{pmatrix}
-2 \\ 2 \\ 11
\end{pmatrix}
+
c_2
\begin{pmatrix}
0 \\ 0 \\ 1
\end{pmatrix}
+
c_3
\begin{pmatrix}
2 \\ 6 \\ -13
\end{pmatrix}
\]
\[
c_1 = -\frac{3}{2}
\wedge
c_2 = \frac{67}{2}
\wedge
c_3 = 1
\]
\[
\vec{x}
=
\frac{67}{2}e^{3t}
\begin{pmatrix}
0 \\ 0 \\ 1
\end{pmatrix}
+ e^{5t}
\begin{pmatrix}
2 \\ 6 \\ -13
\end{pmatrix}
-\frac{3}{2}e^t
\begin{pmatrix}
-2 \\ 2 \\ 11
\end{pmatrix}
\]

\subsection*{6.7}
\subsubsection*{a)}
\[
x'=y 
\;\;\;en\;\;\;
y'=-3y-2x
\]
\[
\begin{vmatrix}
-\lambda & 1\\
-2 & -3-\lambda
\end{vmatrix}
=0
\longrightarrow
\lambda = -1 \vee \lambda = -2
\]
voor $\lambda=-1$:
\[
\left(
\begin{array}{cc|c}
1 & 1 & 0\\
-2 & -2 & 0
\end{array}
\right)
\;\;\;\longrightarrow\;\;\;
\left(
\begin{array}{cc|c}
1 & 1 & 0\\
0 & 0 & 0
\end{array}
\right)
\;\;\;\longrightarrow\;\;\;
c
\begin{pmatrix}
-1 & 1
\end{pmatrix}
\]
voor $\lambda=-2$:
\[
\left(
\begin{array}{cc|c}
2 & 1 & 0\\
-2 & 1 & 0
\end{array}
\right)
\;\;\;\longrightarrow\;\;\;
\left(
\begin{array}{cc|c}
2 & 1 & 0\\
0 & 0 & 0
\end{array}
\right)
\;\;\;\longrightarrow\;\;\;
c
\begin{pmatrix}
-1 & 2
\end{pmatrix}
\]
\[
\vec{x}=
c_1e^{-t}
\begin{pmatrix}
-1\\1
\end{pmatrix}
+ c_2e^{-2t}
\begin{pmatrix}
-1\\2
\end{pmatrix}
\]

\subsection*{6.8}
\subsubsection*{a)}
\[
\vec{x'} = 
\begin{pmatrix}
-2 & 4\\
1 & -4
\end{pmatrix}
\vec{x}
\]
\[
\begin{vmatrix}
-2-\lambda & 4 \\
1 & -4-\lambda
\end{vmatrix}
=0
\;\;\;\longrightarrow\;\;\;
(-2-\lambda)(-4-\lambda)-4=0
\;\;\;\longrightarrow\;\;\;
\lambda = \frac{-6+\sqrt{20}}{2} \;\;\;\vee\;\;\;\lambda = \frac{-6-\sqrt{20}}{2}
\]
Evenwichten:
\[
-2x+4y=0 \;\;\;en\;\;\; -4y+x=0
\]
\[
x=0 \;\;\;en\;\;\; y=0
\]
Stabiliteit:
\[
\lambda_1 < 0 \;\;\;en\;\;\; \lambda_2 < 0
\]
De oorsprong is een stabiel evenwicht.

\subsubsection*{b)}
\[
\begin{vmatrix}
2-\lambda & 4\\
1 & -4-\lambda
\end{vmatrix}
=0
\longrightarrow
\lambda = \frac{-2 \pm \sqrt{52}}{2}
\longrightarrow
instabiel
\]

\subsubsection*{c)}
\[
\begin{vmatrix}
-\lambda & 1\\
-2 & -\lambda
\end{vmatrix}
=0
\longrightarrow
\lambda = \pm i\sqrt{2}
\longrightarrow
geen\;uitspraak\;mogelijk
\]

\subsection*{6.9}
\subsubsection*{a)}
\[
\vec{F}=m.\vec{a}
\]
\[
\begin{pmatrix}
36x\\12x+16y
\end{pmatrix}
=
4
\begin{pmatrix}
x''\\y''
\end{pmatrix}
\longrightarrow
\left\lbrace
\begin{array}{c c}
x''=9\\y''=3x+4y
\end{array}
\right.
\]

\subsubsection*{b)}
\[
\left\lbrace
\begin{array}{c c}
x''=9\\
y''=3x+4y\\
x' = a\\
y' = b
\end{array}
\right.
\longrightarrow
\left\lbrace
\begin{array}{c c}
a=x'\\
b=y'\\
a' = 9x\\
b' = 3x+4y
\end{array}
\right.
\longrightarrow
A=
\begin{pmatrix}
0 & 0 & 1 & 0\\
0 & 0 & 0 & 1\\
9 & 0 & 0 & 0\\
3 & 4 & 0 & 0\\
\end{pmatrix}
\]

\subsubsection*{c)}
\[
\begin{vmatrix}
-\lambda & 0 & 1 & 0\\
0 & -\lambda & 0 & 1\\
9 & 0 & -\lambda & 0\\
3 & 4 & 0 & -\lambda\\
\end{vmatrix}
=0
\longrightarrow
\lambda = \pm 2 \vee \lambda = \pm 3
\]
voor $\lambda=3$:
\[
\left(
\begin{array}{cccc|c}
-3 & 0 & 1 & 0 & 0\\
0 & -3 & 0 & 1 & 0\\
9 & 0 & -3 & 0 & 0\\
3 & 4 & 0 & -3 & 0
\end{array}
\right)
\longrightarrow
\left(
\begin{array}{cccc|c}
1 & 0 & 0 & -\frac{5}{9} & 0\\
0 & 1 & 0 & -\frac{1}{3} & 0\\
0 & 0 & 1 & -\frac{5}{3} & 0\\
0 & 0 & 0 & 0 & 0
\end{array}
\right)
\longrightarrow
c
\begin{pmatrix}
5\\3\\15\\9
\end{pmatrix}
\]
voor $\lambda=-3$:
\[
\left(
\begin{array}{cccc|c}
3 & 0 & 1 & 0 & 0\\
0 & 3 & 0 & 1 & 0\\
9 & 0 & 3 & 0 & 0\\
3 & 4 & 0 & 3 & 0
\end{array}
\right)
\longrightarrow
\left(
\begin{array}{cccc|c}
1 & 0 & 0 & \frac{5}{9} & 0\\
0 & 1 & 0 & \frac{1}{3} & 0\\
0 & 0 & 1 & -\frac{5}{3} & 0\\
0 & 0 & 0 & 0 & 0
\end{array}
\right)
\longrightarrow
c
\begin{pmatrix}
-5\\-3\\15\\9
\end{pmatrix}
\]
voor $\lambda=2$:
\[
\left(
\begin{array}{cccc|c}
-2 & 0 & 1 & 0 & 0\\
0 & -2 & 0 & 1 & 0\\
9 & 0 & -2 & 0 & 0\\
3 & 4 & 0 & -2 & 0
\end{array}
\right)
\longrightarrow
\left(
\begin{array}{cccc|c}
1 & 0 & 0 & 0 & 0\\
0 & 1 & 0 & -\frac{1}{2} & 0\\
0 & 0 & 1 & 0 & 0\\
0 & 0 & 0 & 0 & 0
\end{array}
\right)
\longrightarrow
c
\begin{pmatrix}
0\\1\\0\\2
\end{pmatrix}
\]
voor $\lambda=-2$:
\[
\left(
\begin{array}{cccc|c}
2 & 0 & 1 & 0 & 0\\
0 & 2 & 0 & 1 & 0\\
9 & 0 & 2 & 0 & 0\\
3 & 4 & 0 & 2 & 0
\end{array}
\right)
\longrightarrow
\left(
\begin{array}{cccc|c}
1 & 0 & 0 & 0 & 0\\
0 & 1 & 0 & \frac{1}{2} & 0\\
0 & 0 & 1 & 0 & 0\\
0 & 0 & 0 & 0 & 0
\end{array}
\right)
\longrightarrow
c
\begin{pmatrix}
0\\-1\\0\\2
\end{pmatrix}
\]

\subsection*{6.10}
\subsubsection*{a)}
Evenwichten:
\[
x+xy=0 \;\;\;en\;\;\; 2y-xy=0
\]
\[
x=0 \;\;\;en\;\;\; y=0
\;\;\;\;of\;\;\;\;
x=2 \;\;\;en\;\;\; y=-1
\]
A:
\[
A=
\begin{pmatrix}
1+y & x \\
-y & 2-x
\end{pmatrix}[
\]
voor $(0,0)$:
\[
\begin{pmatrix}
1 & 0 \\
0 & 2
\end{pmatrix}
\]
\[
\lambda_1 = 1
\;\;\;en\;\;\;
\lambda_2 = 2
\]
\[
\lambda_1 \ge 0 \;\;\;of\;\;\; \lambda_2 \ge 0
\]
$(0,0)$ is een onstabiel evenwicht.\\
voor $(-1,2)$:
\[
\begin{pmatrix}
3 & -1 \\
-2 & 3
\end{pmatrix}
\]
\[
\lambda_1 = 7
\;\;\;en\;\;\;
\lambda_2 = -1
\]
\[
\lambda_1 \ge 0 \;\;\;of\;\;\; \lambda_2 \ge 0
\]
$(-1,2)$ is een onstabiel evenwicht.

\subsubsection*{c)}
\[
\left\lbrace
\begin{array}{c c}
x' = x-2xy+xy^2 = 0\\
y' = y+xy = 0
\end{array}
\right.
\]
evenwichten: $(0,0)$ en $(-1,1)$:
\[
A =
\begin{pmatrix}
1-2y+y^2 & -2x+2xy\\
y & 1+x
\end{pmatrix}
\]
voor $(0,0)$:
\[
\begin{pmatrix}
1 & 0 \\
0 & 1 
\end{pmatrix}
\longrightarrow
\lambda=1
\longrightarrow
instabiel
\]
voor $(-1,1)$:
\[
\begin{pmatrix}
0 & 0 \\
1 & 0
\end{pmatrix}
\longrightarrow
geen\;\lambda
\longrightarrow
geen\;uitspraak
\]

\subsection*{6.11}
\subsubsection*{a)}
Lotka Voltura roofdier/prooidier model
\[
x'=x(a-bx+cy)
\;\;\; en \;\;\;
y'=y(-k+lx)
\]
hier:
$a=1$ $b=\frac{1}{4}$ $c=\frac{1}{4}$ $k=1$ $p=1$.
x is roofdier, y is prooidier.

\subsubsection*{b)}
\[
-x+pxy=0 
\;\;\; en \;\;\;
y-\frac{1}{4}y^2-\frac{1}{4} = 0
\]
\[
\longrightarrow (0,0)\;(0,4)\;(\frac{1-4p}{p},\frac{1}{p})
\]

\subsubsection*{c)}
\[
A=
\begin{pmatrix}
-1+py & px\\
\frac{y}{4} & 1-\frac{y}{2}-\frac{x}{4}
\end{pmatrix}
\]
voor $(0,0)$:
\[
\begin{vmatrix}
p-1-\lambda & 0 \\
0 & 1-\lambda
\end{vmatrix}
=0
\longrightarrow \lambda=1 \vee \lambda=p-1
\;\;\; instabiel
\]
voor $(0,4)$:
\[
\begin{vmatrix}
4p-1-\lambda & 0\\
4 & -1-\lambda
\end{vmatrix}
=0
\longrightarrow
\lambda = \frac{-4p-2 \pm \sqrt{16p^2+8p}}{2}
\;\;\; stabiel\;als\;p<\frac{1}{4}
\]
voor $(\frac{1-4p}{p},\frac{1}{p})$
\[
\begin{vmatrix}
0 & 1-4p\\
\frac{1}{4p^2} & 1-\frac{1}{2p}-\frac{1-4p}{4p}
\end{vmatrix}
=0
\longrightarrow
\lambda = \frac{-\frac{3}{4p} \pm \sqrt{\frac{9-64p}{16p^2}}}{2}
\;\;\; stabiel\;als\;p>\frac{1}{4}
\]

\subsection*{13}
\subsubsection*{a)}
\[
\left\lbrace
\begin{array}{c c}
x' = x(a-bx-cy)\\
y' = y(p-qx-ry)
\end{array}
\right.
\]
evenwichten: 
\[
(0,0) (0,\frac{p}{r}) (\frac{a}{b},0) (\frac{cp-ar}{qc-rb},\frac{qa-pb}{qc-rb})
\]

\subsubsection*{b)}
invullen:
\[
(0,0) (0,2) (2,0) (\frac{2}{3},\frac{2}{3})
\]
\[
A = 
\begin{pmatrix}
a-2xb-cy & -cx\\
-qy & p-qxy-2r
\end{pmatrix}
\longrightarrow
\begin{pmatrix}
-2x-2y+2 & -2x\\
-2y & -2x-2y+2
\end{pmatrix}
\]

voor $(0,0)$:
\[
\begin{pmatrix}
2-\lambda & 0 \\
0 & 2-\lambda
\end{pmatrix}
\longrightarrow
instabiel
\]
voor $(0,2)$:
\[
\begin{pmatrix}
-2-\lambda & 0 \\
-4 & -2-\lambda
\end{pmatrix}
\longrightarrow
\lambda = -\frac{1}{2}
\longrightarrow
stabiel
\]
voor $(2,0)$:
\[
\begin{pmatrix}
-2-\lambda & -4 \\
0 & -2-\lambda
\end{pmatrix}
\longrightarrow
\lambda = -\frac{1}{2}
\longrightarrow
stabiel
\]
voor $(\frac{2}{3},\frac{2}{3})$:
\[
\begin{pmatrix}
-\frac{8	}{3}+2 & -\frac{4}{3}\\
-\frac{4}{3} & -\frac{8}{3}+2
\end{pmatrix}
\longrightarrow
\lambda = \frac{2}{3} \vee \lambda=-2
\longrightarrow
instabiel
\]

\section*{Hoofdstuk 7}
\subsection*{7.1}
\subsubsection*{a)}
\[
T_n = 3n+1\;\text{ en }\;T_n=\left\lbrace \begin{array}{c c}
1 &\text{ als }n=0\\
T_{n-1}+3 &\text{ als }n>0
\end{array}\right.
\]

\subsubsection*{b)}
\[
T_n = 3^n\;\text{ en }\;T_n=\left\lbrace \begin{array}{c c}
1 &\text{ als }n=0\\
3T_{n-1} &\text{ als }n>0
\end{array}\right.
\]

\subsubsection*{c)}
\[
T_n = \left( \frac{-1}{5}^n \right)\;\text{ en }\;T_n=\left\lbrace \begin{array}{c c}
1 &\text{ als }n=0\\
\frac{-T_{n-1}}{5} &\text{ als }n>0
\end{array}\right.
\]

\subsection*{7.2}
\subsubsection*{a)}
\[
\begin{array}{c c}
a_1 &= 0\\
a_2 &= \frac{1}{2}\\
a_3 &= 1\\
a_4 &= \frac{3}{2}\\
a_5 &= 2\\
a_6 &= \frac{5}{2}\\
\end{array}
\]

\subsubsection*{b)}
\[
\begin{array}{c c}
b_1 &= 1\\
b_2 &= \frac{2}{3}\\
b_3 &= \frac{4}{9}\\
b_4 &= \frac{8}{27}\\
b_5 &= \frac{16}{81}\\
b_6 &= \frac{32}{243}\\
\end{array}
\]

\subsubsection*{c)}
\[
\begin{array}{c c}
c_1 &= \frac{1}{3}\\
c_2 &= \frac{1}{6}\\
c_3 &= \frac{1}{12}\\
c_4 &= \frac{1}{20}\\
c_5 &= \frac{1}{36}\\
c_6 &= \frac{1}{42}\\
\end{array}
\]

\subsubsection*{d)}
\[
\begin{array}{c c}
u_1 &= 1\\
u_2 &= 1\\
u_3 &= \frac{1}{2}\\
u_4 &= \frac{1}{6}\\
u_5 &= \frac{1}{24}\\
u_6 &= \frac{1}{120}\\
\end{array}
\]

\subsection*{7.7}
$(a_k = b_{k+1}-b_k)$
\[
\frac{1}{k(k+1)} = \frac{1}{k}-\frac{1}{k+1} (b_k = -\frac{1}{k})
\]
\[
\sum_{k=1}^{10}\frac{1}{k(k+1)} = b_{n+1}-b_1 = -\frac{1}{n+1}+1 = \frac{10}{11}
\]

\subsection*{7.8}
\[
\frac{1}{k(k+2)} = \frac{A}{k} + \frac{B}{K+2}=\frac{AK+2A+BK}{k(k+2)}
\]
\[
\left\lbrace\begin{array}{c c}
A+B&=0\\
2A=1\\
\end{array}\right.
\Leftrightarrow
\left\lbrace\begin{array}{c c}
A=\frac{1}{2}\\
B=-\frac{1}{2}\\
\end{array}\right.
\]
\[
\frac{1}{k(k+2)} = \left(\frac{1}{2k}-\frac{1}{2k+4}\right)
\]

\[
\sum_{k=1}^{10}\frac{1}{k(k+2)} = \frac{n(3n+5)}{4(n+1)(n+2)}
\]

\subsection*{7.11}
\subsubsection*{a)}
convergentieïnterval
\[
\sum_{k=0}^{\infty}3^kx^k
\]
\[
L = \lim_{n\rightarrow\infty}\left|\frac{3^{k+1}x^{k+1}}{3^kx^k}\right| = \lim_{n\rightarrow\infty}\left|3x\right|
\]
\[
-1 < 3x < 1 \rightarrow -\frac{1}{3} < x < \frac{1}{3}
\]
\[
x \in \left] -\frac{1}{3},\frac{1}{3} \right[
\]

som $a=1$ en $x=3x$
\[
\frac{1}{1-3x}
\]

\subsubsection*{b)}
\[
\sum_{k=0}^\infty(-1)^kx^{2k}
\]
\[
L 
=
\lim_{n\rightarrow\infty}\left|\frac{(-1)^{k+1}x^{2k+2}}{(-1)^kx^{2k}}\right|
=
|-x^2|
\]
\[
-1 < -x^2 <1 
\;\;\;\longrightarrow\;\;\;
 1>x^2 > -1
 \;\;\;\longrightarrow\;\;\;
 1 > x > -1
\]
\[
x \in \left] -1,1 \right[
\]

som $a=1$ en $x=-x^2$
$$\frac{1}{1+x^2}$$

\subsection*{7.13}
\subsubsection*{a)}
\[
s_n = \frac{1}{1-e^{-\frac{hv}{T}}}
\]
\subsubsection*{b)}
\[
\lim_{T\rightarrow 0+}q(T) = \lim_{T\rightarrow 0+}\sum_{k=0}^\infty e^{-\frac{hkv}{T}}
=1
\]
\[
\lim_{T\rightarrow \infty}q(T) =\lim_{T\rightarrow \infty}\sum_{k=0}^\infty e^{-\frac{hkv}{T}}
=\infty
\]

\subsection*{7.16}
\subsubsection*{b)}
\[
\lim_{k\rightarrow\infty}\ln\left(1+\frac{1}{k}\right)= \ln(1)=0
\]
\[
\sum_{k+1}^\infty \ln\left(1+\frac{1}{n}\right)= \ln(1+1) + \ln\left(1+\frac{1}{2}\right) + \ln\left(1+\frac{1}{3}\right) + ...
\]
\[
=\ln \left( (1+1)(1+\frac{1}{2})(1+\frac{1}{3})...\right)
\]
\[
=\ln \left( 2 (1+\frac{1}{2})(1+\frac{1}{3})...\right)
\]
\[
=\ln \left( 3(1+\frac{1}{3})...\right)
\]
\[
=\ln(k+1)
\]

\subsection*{7.17}
\subsubsection*{a)}
\[
\ln n < n \rightarrow \frac{1}{\ln n} > \frac{1}{n}
\]
\[
0 \le \frac{1}{\ln n} \le \frac{1}{n}
\]
\[
\sum\frac{1}{n}\;is\;divergent \rightarrow \sum\frac{1}{\ln n}\;is\;ook\;divergent
\]
\subsubsection*{b)}
$$\ln j < j \rightarrow \frac{\ln j}{j ^3} < \frac{1}{j ^2} $$
$$\lim_{j\rightarrow\infty}\frac{\frac{\ln j}{j ^3}}{\frac{1}{j ^2}} = \lim_{j\rightarrow\infty}\frac{\ln j}{j} = 0$$
$$\sum_{j=1}^\infty \frac{1}{j^2}  \text{  convergeert naar 1}$$
$$\text{Eig. 7.6.6.:} \sum_{j=1}^\infty \frac{\ln j}{j^3}  \text{  is convergent}$$

\subsection*{7.18}
\subsubsection*{a)}

\[
\lim_{k\rightarrow\infty}\frac{\binom{2k+2}{k+1}}{\binom{2k}{k}}
=
\lim_{k\rightarrow\infty}\frac{\frac{(2k+2)!}{(k+1)!(k+1)!}}{\frac{(2k)!}{k!k!}}
= 
\lim_{k\rightarrow\infty}\frac{(2k+2)!}{(2k)!(k+1)(k+1)}
\]
$$ = \lim_{k\rightarrow\infty}\frac{2(k+1)(2k+1)(2k)!}{(2k)!(k+1)(k+1)}$$
$$ = \lim_{k\rightarrow\infty}\frac{4k + 2}{k + 1} = 4 > 1 \rightarrow \text{divergent}$$

\subsubsection*{b)}
\[
\lim_{k\rightarrow\infty}\frac{\binom{2k+2}{k+1}^{-1}}{\binom{2k}{k}^{-1}}
=
\lim_{k\rightarrow\infty}\frac{\frac{(2k)!}{k!k!}}{\frac{(2k+2)!}{(k+1)!(k+1)!}}
=
\lim_{k\rightarrow\infty}\frac{(2k)!(k+1)(k+1)}{(2k+2)!}
\]
$$= \lim_{k\rightarrow\infty}\frac{(2k)!(k+1)(k+1)}{2(k+1)(2k+1)(2k)!}$$
$$ = \lim_{k\rightarrow\infty}\frac{k + 1}{4k + 2} = \frac{1}{4} < 1 \rightarrow \text{convergent}$$
\subsubsection*{c)}
$$\text{Tip:} \lim_{n\rightarrow\infty}\Big(\frac{n+1}{n}\Big)^n = e $$
\[
\lim_{n\rightarrow\infty}\frac{\frac{(n+1)!}{(n+1)^{n+1}}}{\frac{n!}{n^n}}
=
\lim_{n\rightarrow\infty}\frac{(n+1)!n ^n}{(n+1) ^{n+1} n!}
=
\lim_{n\rightarrow\infty}\frac{(n+1)n ^n}{(n+1)^{n+1}}
\]

$$=\lim_{n\rightarrow\infty}\frac{n^n}{(n+1)^n} = \frac{1}{e^n}$$


\section*{Hoofdstuk 8}
\subsection*{8.1}
\[
R = \lim_{k\rightarrow\infty}\frac{C_k}{C_{k+1}}
\]

\subsubsection*{a)}
\[
\lim_{k\rightarrow\infty}\frac{k^3}{(k+1)^3} = 1
\]

\subsubsection*{b)}
\[
R=\lim_{k\rightarrow\infty}\frac{\binom{2k}{k}}{\binom{2k+2}{k+1}}
=
\lim_{k\rightarrow\infty}\frac{k^2+2k+1}{4k^2+6k+2}
=
\frac{1}{4} 
\]

\subsubsection*{d)}
\[
R=\lim_{n\rightarrow\infty}\frac{\frac{1}{n^2}}{\frac{1}{(n-1)^2}}
=
R=\lim_{n\rightarrow\infty}\frac{(n-1)^2}{n^2}
=
1
\]

\subsection*{8.2}
\subsubsection*{d)}
convergentiestraal:
\[
R=\lim_{k\rightarrow\infty}\frac{2^k}{2^{k+1}}
=
\lim_{k\rightarrow\infty}\frac{1}{2}
=
\frac{1}{2}
\]
som: $a=4x$ en $x=2x$:
\[
\frac{4x}{1-2x}
\]

\subsection*{8.3}
zie voorbeeild 7.2.3 p 133
\[
R=\lim_{k\rightarrow\infty}\frac{\phi_n}{\phi_{n+1}} = \frac{1}{\phi} = \frac{2}{1+\sqrt{5}}
\]

\subsection*{8.5}
\[
\sum_{k=0}^n\frac{f^{(n)}(0)}{k!}x^k
\]
\[
\]

\subsubsection*{a)}
\[
f(x) = \frac{1}{3+0}= \frac{1}{3} \;\;\; f'(x) = \frac{-1}{(3+0)^2}=-\frac{1}{9}
\]
\[
f''(x) = \frac{2}{3+0)^3}=\frac{6}{27} \;\;\; f'''(x) =\frac{-6}{(3+x)^3} = -\frac{6}{81}
\]
dus
\[
\sum_{k=0}^n\frac{\frac{(-1)^k}{3^k}k!}{k!}x^k
=\frac{1}{3}\sum_{k=0}^n(-1)^k\left(\frac{x}{3}\right)^k 
\]

\subsubsection*{c)}
Ga dit niet manueel uitrekenen. Er is een truk voor.
(zie p 158)
\[
\sin(x) = \sum_{k=0}^\infty(-1)^{2k}\frac{x^{2k+1}}{(2k+1)!}
\]
dus
\[
\sin(2x^2) = \sum_{k=0}^\infty(-1)^{2k}\frac{(2x^2)^{2k+1}}{(2k+1)!}
\]

\subsubsection*{d)}
zie c)
\[
e^x = \sum_{k=0}^\infty\frac{x^k}{k!}
\]
dus
\[
e^{-3x} = \sum_{k=0}^\infty\frac{(-3x)^k}{k!}
\]

\subsubsection*{e)}
zie a)
\[
\sin(x) = \sum_{k=0}^\infty(-1)^{2k}\frac{x^{2k+1}}{(2k+1)!}
\]
dus
\[
\frac{\sin(x)}{x} = \frac{\sum_{k=0}^\infty(-1)^{2k}\frac{x^{2k+1}}{(2k+1)!}}{x}
=
\sum_{k=0}^\infty(-1)^{2k}\frac{x^{2k}}{(2k+1)!}
\]

\subsection*{8.7}
\subsubsection*{a)}
\[
\frac{1}{1-x} = \sum_{k=0}^\infty x^k
\]
\[
\frac{d\sum_{k=0}^\infty x^k}{dx} = \sum_{k=0}^\infty kx^{k-1}
= \frac{1}{(1-x)^2}
\]
\[
\sum_{k=0}^\infty kx^{k} = \frac{x}{(1-x)^2} 
\]

\subsubsection*{b)}
neem de afgeleide van het antwoord van a)
\[
\frac{d(\sum_{k=0}^\infty kx^{k})}{dx}
=
\sum_{k=0}^\infty k^2x^{k-1}
=
\frac{d(\frac{x}{(1-x)^2})}{dx}
=
\frac{1+x}{(1-x)^3}
\]
dus
\[
\sum_{k=0}^\infty k^2x^{k}
=
\frac{(1+x)x}{(1-x)^3}
\]

\subsection*{8.10}
Taylorreeks:
\[
\sum_{n=0}^\infty\frac{f(a)^{(n)}}{n!}(x-a)^n
\]

\subsubsection*{a)}
\[
f(x) = \ln(x) \;\; f^{(n)}(x) = (-1)^{(n+1)} \frac{n!}{x^n} \text{ voor } n>0
\]
Taylorreeks:
\[
\ln(a) + \sum_{n=1}^\infty\frac{(-1)^{(n-1)}}{n}\frac{(x-a)}{a}^n
\]

\section*{Hoofdstuk 9}
\subsection*{9.1}
\[
\int_{-l}^l\sin\left(\frac{n\pi x}{l}\right)\sin\left(\frac{m\pi x}{l}\right)dx=0
\]
\[
= \frac{1}{2}\int_{-l}^l\cos\left(\frac{n\pi x - m\pi x}{l}\right)dx - \frac{1}{2}\int_{-l}^l\cos\left(\frac{n\pi x + m\pi x}{l}\right)dx
\]
\[
= \frac{1}{2}\left[\frac{l\sin\left(\frac{n\pi-m\pi}{l}x\right)}{n\pi-m\pi}\right]_{-l}^l
-
\frac{1}{2}\left[\frac{l\sin\left(\frac{n\pi+m\pi}{l}x\right]}{n\pi+m\pi}\right]_{-l}^l
\]
\[
= \frac{1}{2} \left( -\frac{l}{n\pi - m\pi} \left( \sin(n\pi - m\pi)- \sin(m\pi - n\pi) \right) \right)
\]\[
-\frac{1}{2} \left( -\frac{l}{n\pi + m\pi} \left( \sin(n\pi + m\pi)- \sin(-(m\pi + n\pi)) \right) \right)
\]
\[
=0
\]

\subsection*{9.5}
\subsection*{b)}
\[
a_n = \frac{1}{\pi}\int_{-\pi}^\pi f(x)\cos(nx)dx
\]
\[
= \frac{1}{\pi} \int_o^\pi \cos(nx)dx
\]
\[
=\frac{1}{n\pi}\left[\sin(nx)\right]_0^\pi
\]
\[
= \frac{1}{n\pi}(\sin(n\pi)-\sin(0)) = 0
\]\\
\[
b_n = \frac{1}{\pi}\int_{-\pi}^\pi f(x)\sin(nx)dx
\]
\[
= \frac{1}{\pi} \int_0^\pi \sin(nx)dx
\]
\[
= -\frac{1}{n\pi} \left[ \cos(nx) \right]_0^\pi
\]
\[
\frac{1}{n\pi}(\cos(n\pi)-\cos(0))
\]
\[
= \frac{\cos(n\pi)-1}{n\pi}
\]\\
\[
f(x)  = \frac{1}{2} + \frac{2}{\pi} \sum_{n=1}^\infty \frac{1}{2n+1}sin((2n+1)x)
\]

\subsection*{9.6}
\[
\frac{1}{2} + \frac{2}{\pi} \sum_{n=1}^\infty \frac{1}{2n+1}sin((2n+1)x) = \sum_{n=1}^\infty\frac{(-1)^n}{2n+1}
\]
\[
\Longleftrightarrow\; \sin((2n+1)x) = (-1)^n
\]
\[
\Longleftrightarrow\; \left\lbrace \begin{array}{c c}
\sin((2n+1)x) = 1 \text{ als n even}\\ \sin((2n+1)x) = -1 \text{ als n oneven}
\end{array}
\right. \Longleftrightarrow x = \frac{\pi}{2}
\]\\
\[
f(x) = \frac{1}{2} + \frac{2}{\pi} \sum_{n=1}^\infty\frac{(-1)^n}{2n+1}
\]
\[
\sum_{n=1}^\infty\frac{(-1)^n}{2n+1} = (f(\frac{\pi}{2}) - \frac{1}{2})\frac{\pi}{2} = \frac{\pi}{4}
\]

\subsection*{9.8}
\[ b_n =0\]
\[ a_n = \frac{2\omega}{\pi}\int_{\frac{-\pi}{2\omega}}^{\frac{\pi}{2\omega	}} |\sin(\omega x)|\cos(nx)dx
\]
\[
=\frac{2\omega}{\pi}\int_0^{\frac{\pi}{2\omega	}} \sin(\omega x)\cos(nx)dx
\]
\[
=\frac{4}{\pi(1-4n^2)}
\]\\
\[
f(x) = \frac{2}{\pi} - \frac{4}{\pi}\sum_{n=1}^\infty\frac{\cos(2n\omega x)}{4n^2 -1}
\]




\end{document}